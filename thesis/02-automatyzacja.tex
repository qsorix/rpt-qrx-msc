\documentclass[00-praca-magisterska.tex]{subfiles}

\begin{document}

\chapter{Automatyzacja procesu testowania}
\label{automatyzacja}

Idea automatyzacji testów polega na przekazaniu ich wykonywania komputerom.
Celem jest tu oczywiście zaoszczędzenie czasu jaki pochłania praca testerów
ręcznie wykonujących testy.

Koncepcja ta jest bardzo popularna, gdyż możliwość zaoszczędzenia czasu zwykle
oznacza niższe koszty co w każdym projekcie spotyka się z aprobatą. Jak zauważa
James Bach w \cite{snake-oil}, przekonanie o oszczędnościach jest jednak mylne,
a świadomość problemów związanych z automatyzacją jest w branży informatycznej
bardzo niska.

Kierownictwo projektu informatycznego, chcąc wdrożyć automatyczne testy, często
napotyka na trudności i ostatecznie musi się wycofać lub zamiast przewidywanych
oszczędności zaakceptować dodatkowy koszt. Jednym z powodów jest fakt, że
proces automatyzacji wydaje się być prosty, a dopiero w czasie jego
implementacji i adaptacji testów, odkrywa się pełną jego złożoność
\cite{automation-fail}.

Z tych powodów w rozdziale tym przyglądamy się cechom, jakich branża
informatyczna oczekuje od środowisk automatyzujących testy. Mamy nadzieję, że
pozwoli nam to uniknąć najczęściej popełnianych błędów i niedopatrzeń w fazie
projektowej.

\section{Wymagane cechy programu do automatyzacji}
\label{cechy-programu-do-automatyzacji}

Kolejna, po zbagatelizowaniu złożoności problemu, często popełniana pomyłka to
drugorzędowe traktowanie narzędzi do testowania. Automatyzacja procesu wymaga
programu, który będzie to robił.  Program ten musi być rozwijany jak każdy inny
projekt informatyczny, często jednak traktuje się go jako proste i poboczne
narzędzie, które powstanie szybko, w związku z tym nie warto inwestować w jego
długoterminowy rozwój.  Ostatecznie doprowadza to jednak do opóźnień związanych
z brakiem możliwości testowania, na czym cierpi cały projekt.  

MART Pro-Service Solutions w \cite{automation-fail} przedstawia poniższe
wymagania funkcjonalne, jako niezbędne w narzędziu do automatyzacji testów:
\begin{itemize}
\item zapisywanie\footnote{\ang{logging}} przebiegu testu,
\item graficzny kreator scenariuszy testowych,
\item wykonywanie programów rozproszonych,
\item raportowanie,
\item zarządzanie zasobami laboratorium testowego,
\item współbieżne skrypty testowe,
\item przygotowanie i czyszczenie środowiska testowego,
\item wersjonowanie skryptów testowych,
\item dokumentacja.
\end{itemize}

Są one uzupełnione o wymagania pozafunkcjonalne, które często są w projekcie
pomijane:
\begin{itemize}
\item rozszerzalna architektura,
\item możliwość debugowania przebiegu testu,
\item modularne skrypty testowych,
\item orientacja na osiągalne cele projektowe,
\item rozpatrywanie testów dających się łatwo automatyzować.
\end{itemize}

Szczególnie interesujące są dwa ostatnie punkty. Podkreśla się, że nie każdy
test opłaca się automatyzować i należy skupić się na tych, których
automatyzacja jest prosta. W przeciwnym razie ryzykuje się nadmierną
komplikację aplikacji automatyzującej test, co w konsekwencji może uczynić ją
mniej użyteczną.

W artykule \cite{snake-oil} można znaleźć inne zestawienie wymagań
pozafunkcjonalnych, którymi należy się kierować wybierając gotowe narzędzie do
automatyzacji testów. Wymagania te trafnie uzupełniają powyższą listę.
\begin{itemize}
\item użyteczność -- czy narzędzie posiada wymaganą funkcjonalność,
\item niezawodność -- czy narzędzie poprawnie pracuje przez długi okres czasu,
\item łatwość nauki -- czy narzędziem można się szybko zacząć posługiwać,
\item obługa -- czy funkcje narzędzia są łatwo dostępne i użytkownik nie popełni błędu korzystając z nich,
\item wydajność -- czy narzędzie jest wystarczająco szybkie, aby przyspieszyć proces testowania,
\item kompatybilność -- czy narzędzie współpracuje z technologią, jaka ma zostać przetestowana,
\item nieingerencyjność -- jak dobrze narzędzie naśladuje użytkownika.
\end{itemize}

W zależności od zastosowań, program automatyzujący testy często musi dodatkowo
wykonywać takie funkcje jak:
\begin{itemize}
\item wizualizacja przebiegu testu,
\item wizualizacja danych wejściowych,
\item możliwość generowania danych wejściowych,
\item analiza dużych zbiorów danych wynikowych,
\item ocena wyniku.
\end{itemize}

Wszystko to powoduje, że stworzenie lub wybranie idealnego narzędzia do
automatyzacji testów jest trudne i zespół tworzący produkt często musi
dostosować swoje testy do ograniczeń danego rozwiązania.

\section{Koszt automatyzacji}
\label{koszt-automatyzacji}

Testy automatyzuje się w celu obniżenia kosztów. Częstym błędem jest pominięcie
w całkowitym bilansie faktu, że chociaż pewne koszty znikają, w ich miejsce
pojawia się:
\begin{itemize}
\item koszt stworzenia programu i procesu automatyzacji,
\item koszt przeprowadzania automatycznych testów,
\item koszt utrzymywania testów automatycznych wraz ze zmianami w głównym
projekcie.
\end{itemize}

Proces automatyzacji wprowadza również koszt, który związany jest wyłącznie z
automatyzacją i w testowaniu ręcznym nie występuje:
\begin{itemize}
\item przypadki testowe muszą być dokładnie udokumentowane,
\item sam program do automatyzacji oraz skrypty muszą być przetestowane i
udokumentowane,
\item po każdym wykonaniu testów ktoś musi przeanalizować wyniki aby odróżnić
fałszywie zgłoszone defekty\footnote{wynikające np. ze zbyt rygorystycznie
określonego testu} od rzeczywistych błędów,
\item radykalne zmiany w produkcie mogą się wiązać z tworzeniem testów lub
całej automatyzacji od początku,
\item przypadki testowe muszą być przenoszone na inne platformy
sprzętowo-programowe.
\end{itemize}

Kerry Zallar w \cite{practical-automated} zwraca uwagę, że testy wykonywane
automatycznie najlepiej spełniają swoją rolę kiedy mogą działać w dedykowanym
-- czyli też odizolowanym -- środowisku. Tylko wtedy można mieć pewność, że
środowisko to spełnia początkowe warunki testu i nic nie zmieni ich w czasie
jego trwania. Niezależnie od tego czy izolacja ta polega na wydzieleniu czasu,
gdzie testy mają wyłączność na użycie sprzętu, czy też zakup dedykowanych
urządzeń, wciąż wiąże się to z dodatkowym kosztem.

\section{Podsumowanie}
\label{automatyzacja-podsumowanie}

Doświadczenie pokazuje, że najwięcej błędów wykrywa się w czasie
przygotowania i pierwszego wykonania testu. Test, którego warunek raz został
spełniony, rzadko pozwala odkryć nowe błędy. Z tego powodu realny zysk z
automatyzacji ma miejsce tam, gdzie, ze względu np. na ryzyko, opłacalne jest
wielokrotne, ponowne wykonywanie tych samych testów.

Automatyzacja nie eliminuje roli człowieka. Wciąż jest on potrzebny do
tworzenia i utrzymywania testów, a także analizy ich wyników. Osoba obsługująca
testy automatyczne często musi posiadać inne umiejetności niż ta, która
wykonuje je ręcznie. Przykładem może być umiejętność programowania, w celu
tworzenia skryptów testowych.

Ponieważ automatyzacja testów ma zarówno swoje zalety jak i wady, najlepszym
rozwiązaniem jest stosowanie obu rodzajów testów. Oznacza to też, że
automatyzacja najlepiej sprawdza się w pewnym spektrum zastosowań, a próba
wprowadzenia jej w nieodpowiednich przypadkach będzie trudna i nieopłacalna.

\end{document}
