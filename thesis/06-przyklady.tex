\documentclass[00-praca-magisterska.tex]{subfiles}
\begin{document}

\chapter{Przykłady zastosowań Arete}
\label{demonstracja-arete}

\section{Instalacja i uruchomienie}
\label{instalacja-i-uruchomienie-arete}

Instalacja narzędzia Arete jest bardzo prosta dzięki wykorzystaniu wbudowanej w
język Python biblioteki \code{distutils} \cite{distutils}. Polega ona na
wykonaniu poniższej komendy z uprawnieniami administratora.

\code{\# python setup.py install}

Do poprawnego działania wymagane jest również zainstalowanie następujących
zależności:

\begin{itemize}
\item{argparse \cite{argparse} - parsowanie argumentów,}
\item{elixir \cite{elixir} - obsługa bazy danych,}
\item{paramiko \cite{paramiko} - obsługa połączeń SSH}
\end{itemize}

Kolejnym krokiem jest uruchomienie demonów Arete Slave na komputerach biorących
udział w teście. Jeśli uruchamiane przez nas testy będą wymagać uprawnień
administratora musimy wykonać poniższą komendę z takimi uprawnieniami.

\code{\# arete-slave -v}

Ponieważ na każdym z urządzeń będzie działać tylko jeden demon, możemy
skorzystać z domyślnych ustawień. W przeciwnym wypadku wymagane jest użycie
dodatkowych parametrów opisanych w rozdziale \ref{arete-slave-uruchamianie}, aby
zróżnicować porty i nazwy plików bazy danych. Użycie parametru \code{-v}
spowoduje wypisywanie loga na standardowe wyjście, co ułatwi nam obserwację
przebiegu testu. 

Następnie tworzymy konfigurację testu. Po szczegółowy opis tej części
testowania przy pomocy Arete odsyłamy do rozdziału \ref{arete-master-konfiguracja}.
W omówionych w dalszej części tego rozdziału przykładach krok po kroku omówimy
tworzone modele i plany testów. Tymczasem na potrzeby tej instrukcji założymy,
że stworzyliśmy już wspomniane elementy w pliku \code{configuration}.

Niezależnie od tego jaki test chcemy wykonać, potrzebujemy konfiguracji
laboratorium. Zostało to szerzej opisane w rozdziale
\ref{arete-master-laboratorium}. Przykładowy plik laboratorium przedstawiono
poniżej. 

\begin{pythoncode}
  # laboratory:
  create_laboratory('laboratory')

  add_device('marvin',
      connection='tcp',
      ip='192.168.0.2',
      port='4567',
      frontend='arete_slave')

  add_device('defteros',
      connection='tcp',
      ip='192.168.0.3',
      port='4567',
      frontend='arete_slave')
\end{pythoncode}

Kolejnym krokiem jest stworzenie odwzorowania między modelem a laboratorium.
Odwzorowania były omówione w rozdziale \ref{arete-master-odwzorowanie}. 
Jeśli chcielibyśmy zawrzeć je w pliku konfiguracyjnym, wyglądałoby to
następująco.

\begin{pythoncode}
  create_mapping('configuration-to-laboratory')

  bind('host1', 'marvin')
  bind('host2', 'defteros')
\end{pythoncode}

Alternatywnie, odwzorowanie możemy podać w linii poleceń programu Arete Master,
korzystając z argumentu \code{--map}, jak przedstawiono w przykładzie poniżej.
Test uruchamiamy podając wszystkie pliki konfiguracyjne po argumencie \code{-c}.
Opis wszystkich parametrów znajduje się w rozdziale \ref{arete-master-uruchamianie}.

\code{\$ arete -c configuration laboratory --map host1:marvin host2:defteros}

\section{Przykład: Aplikacja klient-serwer}
\label{example-client-server}

Pierwszym przykładem, który omówimy jest prosty scenariusz, z którym często mamy
do czynienia w czasie tworzenia oprogramowania typu klient-serwer. W naszym
przypadku będzie to program umożliwiający transfer plików, gdzie zarówno serwer
jak i klient będą sterowane z linii poleceń. W teście będziemy chcieli sprawdzić
czy transfer przebiegł poprawnie, tj. czy zawartość przesłanego pliku jest
identyczna i czy zgadza się jego nazwa.

W przykładzie wykorzystaliśmy proste aplikacje napisane w języku Python, których
kod zamieszczamy poniżej. Klient łączy się z serwerem i korzystając z protokołu
TCP przesyła nazwę i zawartość wskazanego pliku. Serwer nasłuchuje na
przychodzące połączenia i odbiera nazwę oraz zawartość transmitowanego pliku, a
następnie zapisuje go w lokalnym systemie plików.

\begin{pythoncode}
  # Client.py:
  import sys
  import socket

  if __name__ == "__main__":
      if len(sys.argv) != 5:
	  print >>sys.stderr, \
	      "Usage: {0} <ip> <port> <file> <destname>".format(sys.argv[0])
	  sys.exit(1)

      ip = sys.argv[1]
      port = int(sys.argv[2])
      filename = sys.argv[3]
      destname = sys.argv[4]

      sock = socket.socket(socket.AF_INET, socket.SOCK_STREAM)
      sock.connect((ip, port))

      sock.send(destname+'\n')

      with open(filename, 'r') as file:
          sock.sendall(file.read())

      sock.close()

      print 'transfer done'

\end{pythoncode}

\begin{pythoncode}
  # Server.py:
  import sys
  import SocketServer

  class FileTransferHandler(SocketServer.BaseRequestHandler):
      def handle(self):
          input  = self.request.makefile('r', 0)
          filename = input.readline().strip()

          print 'received', filename

          with open(filename, 'w') as file:
	      file.write(input.read())
  
          self.request.close()

  if __name__ == "__main__":
      if len(sys.argv) != 3:
          print >>sys.stderr, "Usage: {0} <ip> <port>".format(sys.argv[0])
          sys.exit(1)

      ip = sys.argv[1]
      port = int(sys.argv[2])

      server = SocketServer.TCPServer((ip, port), FileTransferHandler)
      server.serve_forever()

\end{pythoncode}

\subsection{Przebieg testu}

\begin{figure}[htb]
\begin{center}
\leavevmode
\includegraphics[width=0.47\textwidth]{example-01-sequence}
\end{center}
\caption{Schemat przebiegu testu aplikacji klient-serwer.}
\label{fig:example-01-sequence}
\end{figure}

Zacznijmy od prześledzenia zobrazowanego na rysunku \ref{fig:example-01-sequence}
przebieg testu. Zaczynamy od wystartowania serwera (\code{1}) na urządzeniu
\code{Server}.  Następnie na drugim urządzeniu uruchamiany jest klient
(\code{2}), który łączy się z serwerem i przesyła do niego wskazany plik. Po
zakończeniu transferu klient kończy działanie i uruchamiane jest obliczenie sumy
kontrolnej (\code{3}) oraz wysyłane jest powiadomienie (\code{4}) do modułu
Arete Master. Ten wysyła wyzwalacz do Arete Slave na urządzeniu \code{Server},
który odbiera go (\code{5}) i kończy pracę serwera. Na koniec na urządzeniu
\code{Server} sprawdzana jest nazwa pliku (\code{6}) oraz suma kontrolna
(\code{7}), a następnie przesłany plik jest usuwany (\code{8}).

\subsection{Konfiguracja}

Poniżej omówiliśmy szczegółowo konfigurację testu, w ramach którego zostaną
wykonane opisane przed chwilą czynności. Zawiera ona jedynie model i plan testu.
Zakładamy, że użytkownik zapoznał się z rozdziałami \ref{arete-master-konfiguracja}
oraz \ref{instalacja-i-uruchomienie-arete} omawiającymi konfigurację oraz uruchomienie
testów za pomocą Arete.

\begin{pythoncode}
  # examples/client-server/configuration:
  create_model('client-server')

  add_host('client')
  add_host('server')
\end{pythoncode}

Na początku tworzymy model o nazwie \code{client-server}. Ponieważ do
przeprowadzenia testu potrzebujemy dwóch urządzeń - klienta i serwer, dodajemy
je do stworzonego modelu. 

\begin{pythoncode}
  create_schedule('client-server')

  test_end_policy('complete')
  create_trigger('kill-server', 1)
\end{pythoncode}

Tworzymy również plan testu \code{client-server}, aby móc określić jakie
komendy będą wykonywane na urządzeniach, a następnie konfigurujemy globalne
ustawienia testu. Ponieważ nie wiemy jak długo potrwa test, ustawiamy politykę
zakończenia testu na \code{complete}. Spowoduje to zakończenie testu dopiero w
momencie, gdy wykonają się komendy na wszystkich urządzeniach biorących udział
w teście.  Zwróćmy uwagę, że jednym z urządzeń jest serwer, który z założenia
nie kończy swojego działania. Z tego powodu tworzymy wyzwalacz o nazwie
\code{kill-server}, mający za zadanie zakończyć działanie serwera, gdy
określona ilość urządzeń (w naszym przypadku jeden klient), powiadomi o
zakończeniu testu. Wartość tę podajemy jako drugi parametr komendy
\code{create\_trigger}.

\begin{pythoncode}
  add_resource(File('clientdist', 'Client.py', chmod='+x'))
  add_resource(File('serverdist', 'Server.py', chmod='+x'))
  add_resource(File('testfile', 'datafile.txt'))
\end{pythoncode}

Kolejnym etapem jest dodanie zasobów biorących udział w teście. Będą to
oczywiście programy \code{Client.py} oraz \code{Server.py}. Aby zapewnić im
odpowiednie prawa po przesłaniu na zdalne urządzenia, koszystamy z parametru
\code{chmod}. Jako trzeci dodajemy plik, który zostanie przesłany między
urządzeniami.

\begin{pythoncode}
  append_schedule('client', [
      ('client', at(1), \
          shell('./@{clientdist.name} @{server.ip} 6666 \
	  @{testfile.name} dest_path', \
	  use_resources=['clientdist', 'testfile'], check_executable=False)),
      ('md5sum', after('client'), shell('md5sum @{testfile.name}')),
      ('notify', after('md5sum'), notify('kill-server'))
  ])
\end{pythoncode}

Najważniejszą częścią konfiguracji jest określenie szczegółowego planu
wykonywania komend na urządzeniach. W przypadku klienta uruchamiamy program
\code{Client.py} z odpowiednimi parametrami, tj. adresem i portem serwera, nazwą
pliku który zostanie przesłany oraz docelową nazwą pliku. W parametrze
\code{use\_resources} podajemy nazwy stworzonych wcześniej zasobów. Parametr
\code{check\_executable=False} zapobiega sprawdzeniu czy komenda którą chcemy
wykonać jest wykonywalna, ponieważ sami zadbaliśmy o to nadając plikom
odpowiednie prawa podczas ich tworzenia. Zwróćmy uwagę, że komenda \code{client}
uruchamiana jest z sekundowym opóźnieniem (\code{at(1)}), aby pozostawić czas na
uruchomienie serwera.

Następnie dodajemy dwa zadania, które będą wykonywane gdy program klienta
zakończy działanie (funkcja \code{after()}). Pierwsze z nich to komenda
sprawdzająca sumę kontrolną przesyłanego pliku korzystając z polecenia
\code{md5sum}. Drugie powoduje powiadomienie Arete Master o zakończeniu
działania. Ponieważ wyzwalacz o nazwie \code{kill-server} czeka tylko na
jedno powiadomienie, Arete Slave otrzyma dzięki niemu sygnał do wykonania
komendy kończącej działanie serwera.

\begin{pythoncode}
  append_schedule('server', [
      ('server', at(0), \
          shell('./@{serverdist.name} 0.0.0.0 6666', \
          use_resources=['serverdist'], check_executable=False)),
      ('trigger', trigger('kill-server'), shell('kill @{server.pid}')),
      ('ls', after('server'), shell('ls dest_path')),
      ('md5sum', after('ls'), shell('md5sum dest_path')),
      ('rm', after('md5sum'), shell('rm -f dest_path'))
  ])
\end{pythoncode}

W przypadku serwera na początku uruchamiany jest program \code{Server.py} z
odpowiednimi parametrami. Podobnie jak w kliencie przesyłane są wymagane zasoby
oraz flaga \code{check\_executable} ustawiana jest na \code{False}. Następnie
tworzona jest komenda kończąca działanie komendy \code{server}, która zostanie
uruchomiona po otrzymaniu sygnału o nazwie \code{kill-server} od modułu Arete
Master. Pozwoli to na uruchomienie dalszych komend, sprawdzających nazwę oraz
sumę kontrolną otrzymanego od klienta pliku, usunięcie go oraz zakończenie
testu.

\subsection{Wyniki}

Po zakończeniu testu Arete Master odbiera wyniki od obu modułów Slave
uczestniczących w teście i zapisuje je do swojej bazy danych. W tym przykładzie
wyglądają one następująco.

\begin{description}
\item[]
  node: client
  \begin{description}
  \item[start time] 2010-11-14 21:46:46.171340
  \item[duration] 1.442621
  \item[]
    command: 'client'
    \begin{description}
      \item[output] transfer done
      \item[start time] 2010-11-14 21:46:47.171994
      \item[duration] 0.049891
      \item[return code] 0
    \end{description}
  \item[]
    command: 'md5sum'
    \begin{description}
      \item[output] 015f88fc887a556bf7eab90ec32c8c46  testfile
      \item[start time] 2010-11-14 21:46:47.371130
      \item[duration] 0.057670
      \item[return code] 0
    \end{description}
  \end{description}
\end{description}

\begin{description}
\item[]
  node: server
  \begin{description}
  \item[start time] 2010-11-14 21:46:46.172080
  \item[duration] 1.802907
  \item[]
    command: 'server'
    \begin{description}
      \item[output] 
      \item[start time] 2010-11-14 21:46:46.179522
      \item[duration] 1.532876
      \item[return code] -15
    \end{description}
  \item[]
    command: 'ls'
    \begin{description}
      \item[output] dest\_path
      \item[start time] 2010-11-14 21:46:47.733533
      \item[duration] 0.056964
      \item[return code] 0
    \end{description}
  \item[]
    command: 'md5sum'
    \begin{description}
      \item[output] 015f88fc887a556bf7eab90ec32c8c46  dest\_path
      \item[start time] 2010-11-14 21:46:47.823826
      \item[duration] 0.049728
      \item[return code] 0
    \end{description}
  \end{description}
\end{description}

Jak widzimy w bazie danych oprócz informacji dotyczących całego testu, takich
jak czas jego rozpoczęcia i długość trwania na każdym z komputerów, mamy do
dyspozycji szczegółowe dane na temat każdej z wykonanych komend, tj. wyjście,
zwrócony kod wyjścia, czas rozpoczęcia oraz długość trwania.

Choć interpretacja wyników nie jest celem naszej pracy, na potrzeby tego prostego
przykładu możemy zaobserwować, że nazwa pliku otrzymana z komendy ls na serwerze
jest poprawna, a sumy kontrolne na obu urządzeniach są zgodne. Możemy zatem
wywnioskować, że stworzone przez nas implementacje programów klient-serwer są
poprawne.

\section{Przykład: Wpływ transmisji TCP na pracę DCCP}
\label{example-dccp}

Drugi przykład dotyczy pomiaru szybkości transferu danych przy użyciu protokołu
DCCP oraz wpływu innych transmisji na tę szybkość. Sposób pomiaru jest prosty.
Wykorzystujemy program \code{iperf}, który umożliwia uruchomienie serwera i
klienta wybranego protokołu. Klient transmituje do serwera strumień losowych
danych korzystając ze wskazanego protokołu transportowego i zapamiętuje uzyskaną
szybkość transmisji. Dodatkowo, przy użyciu reguł \code{iptables} mierzy się
ilość pakietów i bajtów całego ruchu sieciowego.

Stworzyliśmy kilka różnych testów badających m.in.: samodzielną pracę DCCP,
wpływ transmisji TCP i UDP na pracę DCCP, porównanie DCCP z TCP i UDP, itd.
Różniły się one kolejnością uruchamiania oraz długością działania programu
\code{iperf} dla poszczególnych protokołów. W tym przykładzie zaprezentujemy
test przedstawiający wpływ transmisji TCP na pracę DCCP. 

\subsection{Przebieg testu}

\begin{figure}[htb]
\begin{center}
\leavevmode
\includegraphics[width=0.56\textwidth]{example-03-sequence}
\end{center}
\caption{Schemat przebiegu testu badającego wpływ transmisji TCP na pracę
protokołu DCCP.}
\label{fig:example-03-sequence}
\end{figure}

Przebieg testu przestawiliśmy na rysunku \ref{fig:example-03-sequence}. Na
początku na urządzeniu \code{server} uruchamiane są dwa serwery \code{iperf} -
jeden dla protokołu TCP, drugi korzystający z DCCP. Następnie na obu
urządzeniach rozpoczynamy wykonywanie komendy \code{iptables} zwracającej
statystyki ruchu sieciowego. Komenda ta będzie wykonywana co sekundę aż do
zakończenia testu. Z sekundowym opóźnieniem uruchamiamy na urządzeniu
\code{client} klienta \code{iperf}, który rozpoczyna komunikację z odpowiednim
serwerem z wykorzystaniem protokołu DCCP. W 10 sekundzie rozpoczynamy transmisję
TCP uruchamiając kolejnego klienta \code{iperf}. Test kończy się po 20
sekundach.

\subsection{Konfiguracja}

\begin{pythoncode}
  class IPTablesCounters(HostDriverPlugin):
      def process(self, cmd, host, attributes):
          if 'iptables_counters' not in attributes:
              return 

          protocols = host.model['iptables_counters']

          cmd.add_setup('iptables -F')
          for proto in protocols:
              cmd.add_setup('iptables -I INPUT -p {0}'.format(proto))
              cmd.add_setup('iptables -I OUTPUT -p {0}'.format(proto))

          cmd.add_cleanup('iptables -F')

          attributes.remove('iptables_counters')
\end{pythoncode}

Zaczynamy od stworzenia sterownika parametrów (\code{HostDriverPlugin} - opisany
szerzej w rozdziale \ref{arete-master-sterowniki-parametrow}), który dla każdego z urządzeń doda
komendy konfiguracyjne i czyszczące tworzące odpowiednie wpisy w \code{iptables}
dla podanych protokołów.

\begin{pythoncode}
  create_model('dccp_flow')

  add_host('client', iptables_counters=['33', 'tcp', 'udp'])
  add_host('server', iptables_counters=['33', 'tcp', 'udp'])
\end{pythoncode}

Następnie tworzymy model o nazwie \code{dccp\_flow} i podobnie jak w poprzednim
przykładzie dodajemy dwa urządzenia - klienta i serwer. W parametrze
\code{iptables\_counters} podajemy listę protokołów dla wcześniej stworzonego
sterownika parametrów. Liczba \code{33} odpowiada protokołowi DCCP.

\begin{pythoncode}
  create_schedule('dccp_flow')

  test_end_policy('duration 20', setup_phase_delay=2.0)
\end{pythoncode}

Tworzymy także plan testu o nazwie \code{dccp\_flow}. Ponieważ ważne jest, żeby
w czasie testu program nie przesyłał żadnych nadmiarowych danych, które
wpłynęłyby na uzyskane wyniki, ustawiamy czas trwania testu na stałą wartość
\code{20} sekund. Spowoduje to rozłączenie się Arete Master po przesłaniu testu
i ponowne połączenie po zakończeniu testu. Ponieważ test wymaga wykonania komend
konfiguracyjnych ustawiamy opóźnienie testu na \code{2.0} korzystając z
parametru \code{setup\_phase\_delay}.

\begin{pythoncode}
  dccp = ClientServer('dccp',
      server_command='iperf -s -p 7999 -d',
      client_command='iperf -c @{server.ip} -p 7999 -d -t 18')

  tcp = ClientServer('tcp',
      server_command='iperf -s -p 7998',
      client_command='iperf -c @{server.ip} -p 7998 -t 9')
\end{pythoncode}

Korzystając z gotowej klasy \code{ClientServer} z modułu \code{utils} tworzymy
komendy uruchamiające \code{iperf} w trybie klienta (\code{-c}) oraz serwera
(\code{-s}) dla protokołów DCCP (opcja \code{-d}) oraz TCP (domyślnie).

\begin{pythoncode}
  append_schedule('server', dccp.server(start=0, end=20))
  append_schedule('client', dccp.client(start=1, end=None, server='server'))

  append_schedule('server', tcp.server(start=0, end=20))
  append_schedule('client', tcp.client(start=10, end=None, server='server'))

  append_schedule('server', [('counters', every(1), \
      shell('iptables -L -v -n -Z -x'))])
  append_schedule('client', [('counters', every(1), \
      shell('iptables -L -v -n -Z -x'))])
\end{pythoncode}

Na koniec dodajemy do planu testu stworzone wcześniej komendy. Na urządzeniu
\code{server} dodajemy serwery DCCP i TCP, które działać będą przez cały czas
trwania testu. Na urządzeniu \code{client} w \code{1} sekundzie uruchamiamy
klienta DCCP, a w \code{10} sekundzie klienta TCP. Na obu urządzeniach co
sekundę zapisujemy wynik komendy wypisującej ilość przesłanych pakietów i
bajtów.

\subsection{Wyniki}

Korzystając z otrzymanych wyników w postaci statystyk zebranych
przez cosekundowe wywołania komendy \code{iptables} byliśmy w stanie sporządzić
wykresy, takie jak ten przedstawiony na rysunku \ref{fig:tcp-dccp}. Istotny jest
tu fakt, że stworzenie kolejnych wykresów dla różnych sytuacji polega jedynie
na prostych zmianach w konfiguracji testu.

\begin{figure}[htb]
\begin{center}
\leavevmode
\includegraphics[width=0.8\textwidth]{tcp-dccp}
\end{center}
\caption{Wykres przedstawiający wpływ transmisji TCP na pracę
protokołu DCCP.}
\label{fig:tcp-dccp}
\end{figure}

\section{Przykład: Dystrybucja pliku w sieci BitTorrent}
\label{example-torrent}

Trzeci, ostatni przykład, który przedstawimy, to ćwiczenie laboratoryjne z
Systemów Peer-to-Peer. Test polega na pomiarze czasu dystrybucji pliku w sieci
BitTorrent. Klienci mają rozpocząć pobieranie w tym samym momencie, a test należy
zakończyć, kiedy wszyscy będą posiadać pełny plik.

%Interesuje nas, ile każdy
%klient wysłał danych i ile trwała pełna dystrybucja pliku.

Przykład ten można wykorzystać do stworzenia testów na różnych topologiach,
poniżej omówimy sytuację z dowolną ilością urządzeń znajdujących się w tej samej
podsieci. Oprócz urządzeń typu peer\footnote{\emph{ang. pośrednik} - użytkownik,
który w danym momencie pobiera lub udostępnia dany plik.} wykorzystamy dwa
osobne urządzenia, które będą pełnić rolę trackera\footnote{\emph{ang.
tropiciel} - serwer przekazujący informacje użytkownikach pobierających dany
plik.} i seeda\footnote{\emph{ang. obsiewacz} - użytkownik, który posiada
kompletny plik i udostępnia go innym osobom.}.

Do stworzenia danych, które będą przesyłane w czasie testu oraz pliku
torrent\footnote{Metaplik zawierający niezbędne informacje do rozpoczęcia
pobierania pliku, takich jak zawartość archiwum i adres trackera czy sumy
kontrolne plików.} można wykorzystać stworzony przez nas skrypt
\code{create\_torrent.sh} znajdujący się w katalogu z przykładem. Pobiera on
trzy argumenty - adres i port, na których docelowo będzie uruchomiony tracker
oraz rozmiar tworzonego pliku danych.

\subsection{Przebieg testu}

\begin{figure}[htb]
\begin{center}
\leavevmode
\includegraphics[width=0.7\textwidth]{example-02-sequence}
\end{center}
\caption{Schemat przebiegu testu z wykorzystaniem sieci BitTorrent.}
\label{fig:example-02-sequence}
\end{figure}

Zacznijmy od opisu przebiegu testu, który przedstawiono na rysunku
\ref{fig:example-02-sequence}. Na urządzeniu \code{tracker} uruchamiamy tracker protokołu
BitTorrent, a następnie na urządzeniu \code{seed} rozpoczynamy udostępnianie
pliku przez jednego z klientów (\emph{ang. seeding}). Z parosekundowym
opóźnieniem, równocześnie uruchamiamy na urządzeniach \code{peer} klientów,
którzy pobierają plik. Gdy klient kończy pobieranie wysyła powiadomienie do
modułu Arete Master. W momencie gdy ten otrzyma powiadomienia od wszystkich
klientów działających na urządzeniach \code{peer} wysyła wyzwalacze do Arete
Slave na wszystkich urządzeniach. Powoduje to zakończenie pracy trackera oraz
wszystkich klientów i zakończenie testu.

\subsection{Konfiguracja}

\begin{pythoncode}
  # examples/torrent/model_x_peers
  create_model('torrent')

  tracker = add_host('tracker')
  seed = add_host('seed')

  peers = []
  for i in range(10):
      peers.append(add_host('peer{0}'.format(i)))
\end{pythoncode}

Rozpoczynamy od stworzenia modelu o nazwie \code{torrent}. Dodajemy do niego dwa
urządzenia, które będą pełnić rolę trackera i seeda, a następnie określoną ilość
(w tym wypadku 10) pozostałych urządzeń które będę brały udział w teście.

\begin{pythoncode}
  # examples/torrent/schedule_torrent
  create_schedule('torrent')
    
  test_end_policy('complete')
  create_trigger('stop-torrent', len(peers))
\end{pythoncode}

Następnie przechodzimy do stworzenia planu testu \code{torrent}, określając jego
politykę zakończenia na \code{complete}. Tworzymy również wyzwalacz o nazwie
\code{stop-torrent}, ustawiając jego wartość na ilość urządzeń typu \code{peer}
biorących udział w teście.

\begin{pythoncode}
  torrent = add_resource(File('torrent', 'big-data-file.torrent'))
  data = add_resource(File('torrent-data', 'big-data-file'))
  trackcfg = add_resource(File('trackcfg', 'xbt_tracker.conf'))
\end{pythoncode}

Kolejnym etapem jest stworzenie zasobów. Dodajemy stworzony wcześniej plik z
danymi oraz plik torrent. Dodatkowo przesyłamy plik konfiguracyjny dla trackera
XBT.

\begin{pythoncode}
  tracker_schedule = [
      ('tracker', at(0), shell('xbt_tracker --conf_file @{trackcfg.name}', \
          use_resources=[trackcfg])),
      ('stop-torrent', trigger('stop-torrent'), shell('kill @{tracker.pid}'))
  ]
\end{pythoncode}

Szczegółowy plan testu rozpoczynamy od urządzenia \code{tracker}. Na początku testu
uruchamiamy program \code{xbt\_tracker} podając jako argument przesłany z
wykorzystaniem parametru \code{use\_resources}, stworzony wcześniej plik
konfiguracyjny. Na koniec dodajemy komendę kończącą działanie trackera, która
zostanie uruchomiona po otrzymaniu sygnału o nazwie \code{stop-torrent} od
modułu Arete Master.

\begin{pythoncode}
  seed_schedule = [ 
      ('ctorrent', at(1), shell('ctorrent "@{torrent.name}"', \
          use_resources=[torrent, data])),
      ('stop', trigger('stop-torrent'), shell('kill -9 @{ctorrent.pid}'))
  ]
\end{pythoncode}

Następnie tworzymy plan testu dla urządzenia udostępniającego. Dodajemy komendę
uruchamiającą klienta \code{ctorrent} oraz kończącą go w odpowiedzi na wyzwalacz
\code{stop-torrent}. W parametrze \code{use\_resources} podajemy zarówno plik
torrent jak i plik z danymi które mają być udostępniane.

\begin{pythoncode}
  peer_schedule = [
      ('ctorrent', at(2), \
          shell('ctorrent "@{torrent.name}" -X @{poke stop-torrent}', \
          use_resources=[torrent])),
      ('poke', poke('stop-torrent'), notify('stop-torrent')),
      ('stop', trigger('stop-torrent'), shell('kill -9 @{ctorrent.pid}'))
  ]
\end{pythoncode}

Podobnie jak w przypadku urządzenia \code{seed}, plan testu na pozostałych
urządzeniach rozpoczyna się od uruchomienia klienta \code{ctorrent}. Różnica
polega na tym, że jest on uruchamiany dopiero w drugiej sekundzie testu oraz
nie posiada pliku danych, gdyż ten nie jest przesyłany w parametrze
\code{use\_resources}.

Ponieważ chcemy aby klient po odebraniu pliku nadal udostępniał go innym, a
jednocześnie chcemy poznać, kiedy każdy zakończy pobieranie, wykorzystujemy
mechanizm powiadomień. Klient \code{ctorrent} pozwala przy pomocy argumentu
\code{-X} przekazać polecenie do wykonania po zakończeniu pobierania pliku.
Przekazana w ten sposób komenda \code{\@\{poke stop-torrent\}} uruchamia program
\code{arete-poker} z odpowiednimi argumentami, który łączy się z Arete Slave i
uruchamia polecenie \code{poke}, które z kolei wysyła powiadomienie do modułu
Arete Master. Gdy moduł Master odbierze wszystkie powiadomienia, uruchomi
wyzwalacz \code{stop-torrent}, który spowoduje zamknięcie klientów
\code{ctorrent} i zakończenie testu.

\begin{pythoncode}
  append_schedule('tracker', tracker_schedule)
  append_schedule('seed', seed_schedule)

  for host in peers:
      append_schedule(host, peer_schedule)
\end{pythoncode}

Na koniec dodajemy odpowiednie plany do stworzonych w modelu urządzeń.

%\subsection{Wyniki}

%\FIXME{Wrzucić wyniki testu i krótko je opisać.}

\section{Podsumowanie}
\label{przyklady-podsumowanie}

Opisane w tym rozdziale przykłady pokazują jak łatwo i efektywnie tworzyć
testy korzystając ze stworzonego przez nas narzędzia. Zarówno w prostych 
przypadkach jak testowanie aplikacji klient-serwer, jak i przy testach 
protokołu BitTorrent na wielu komputerach, Arete zachowuje swoją prostotę, 
pozwalając oszczęścić czas i skupić się na interesujących nas wynikach.

Największą zaletą przy testowaniu z wykorzystaniem Arete jest łatwość z 
jaką możemy przeprowadzać podobne do siebie testy. Niewielkie zmiany w 
scenariuszu testu wymagają tu jedynie niewielkich zmian w konfiguracji. 
Wszystko to przy zachowaniu raz stworzonego schematu laboratorium oraz 
dynamicznemu odwzorowaniu urządzeń biorących udział w teście na fizyczne 
maszyny.

\end{document}
