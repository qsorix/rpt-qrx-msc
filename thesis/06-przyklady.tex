\documentclass[00-praca-magisterska.tex]{subfiles}
\begin{document}

\chapter{Przykłady}

\FIXME{Na razie jest tu tylko kopia tekstu, który opisywał przykładowe
scenariusze. Tekst w pierwotnym miejscu został streszczony, a tutaj jest jego
oryginalna postać. Trzeba ją pociąć do odpowiednich podrozdziałów.}

Najprostszy scenariusz ma miejsce w czasie tworzenia oprogramowania typu
klient-serwer. Załóżmy, że tworzymy prosty program mający umożliwiać transfer
plików. Zarówno serwer jak i klient będzie sterowany z linii poleceń. W teście
chcemy sprawdzić czy transfer przebiegł poprawnie, tj. czy zgadza się z nazwa
pliku i czy jego zawartość jest identyczna.

Klient łączy się z serwerem i korzystając z protokołu TCP przesyła nazwę i
zawartość wskazanego pliku. Serwer nasłuchuje na przychodzące połączenia i
odbiera transmitowany plik, który zapisuje w lokalnym systemie plików.

Aby wykonać pomiar musimy zbudować aktualną wersję kodu źródłowego i
przygotować pliki wykonywalne serwera do transmisji. Odpowiedni plik (np. Java
Archive) powinien wygenerować używany przez nas skrypt budujący.

Gotowe pliki trzeba przesłać na maszynę docelową i uruchomić serwer odpowiednim
poleceniem. Pliki aplikacji klienckiej będą uruchamiane lokalnie.

Aby sprawdzić poprawność tranmisji, po jej zakończeniu na obu systemach
wykonujemy polecenia \code{ls} i \code{md5sum}.

W idealnej sytuacji, wszystkie opisane wyżej czynności będą wymagały wydania
tylko dwóch poleceń.

Zadanie narzędzia polega tutaj na:
\begin{enumerate}
\item przesłaniu aktualnej implementacji serwera na wyznaczone urządzenie,
\item uruchomieniu serwera i klienta z odpowiednimi parametrami,
\item zakończeniu serwera po wykonaniu testu,
\item obliczeniu sum kontrolnych plików,
\item zapamiętaniu wyników z obu komponentów,
\item usunięcie pliku ze zdalnej maszyny.
\end{enumerate}

Drugi scenariusz dotyczy pomiaru szybkości transferu danych przy użyciu
protokołu DCCP (Datagram Congestion Control Protocol) oraz wpływu innych
transmisji na tę szybkość. Sposób pomiaru jest prosty. Wykorzystujemy program
\code{iperf}, który umożliwia uruchomienie serwera i klienta wybranego
protokołu. Klient transmituje do serwera strumień losowych danych korzystając
ze wskazanego protokołu transportowego i zapamiętuje uzyskaną szybkość
transmisji. Dodatkowo, przy użyciu reguł \code{iptables} mierzy się ilość
pakietów i bajtów całego ruchu sieciowego.

Stworzyliśmy kilka różnych testów badających m.in.: samodzielną pracę DCCP,
wpływ transmisji TCP i UDP na pracę DCCP, porównanie DCCP z TCP i UDP, itd.
Różniły się one kolejnością i momentami, w których na uczestniczących w teście
urządzeniach należało uruchomić lub zakończyć \code{iperf}.

Aby zautomatyzować te testy, program musi:
\begin{enumerate}
\item korzystając z interfejsu \code{sysctl} skonfigurować parametry DCCP,
\item stworzyć reguły \code{iptables} do mierzenia ruchu,
\item przeprowadzić synchronizację, 
\item rozpocząć test,
\item w wyznaczonych momentach rozpocząć lub zakończyć działanie \code{iperf},
\item zbierać wartości liczników \code{iptables},
\item zapamiętać wszystkie uzyskiwane wyniki,
\item przywrócić opcje DCCP i konfigurację \code{iptables}.
\end{enumerate}

Ważne jest też, żeby w czasie testu program nie przesyłał żadnych danych, aby
nie wpływał na uzyskiwane wyniki.

Trzeci scenariusz, który mieliśmy na uwadze, to ćwiczenie laboratoryjne z
Systemów Peer-to-Peer. Test polega na pomiarze czasu dystrybucji pliku w sieci
BitTorrent. Klienci mają rozpocząć pobieranie w tym samym momencie, a test należy
zakończyć kiedy wszyscy będą posiadać pełny plik. Interesuje nas, ile każdy
klient wysłał danych i ile trwała pełna dystrybucja pliku. Zadana topologia
urządzeń to dwie sieci IP po cztery urządzenia, połączone routerem sztucznie
generującym opóźnienia.

Program w tym teście:
\begin{enumerate}
\item do każdego uczestnika wysyła plik torrent,
\item konfiguruje interfejsy routera,
\item konfiguruje adresy sieciowe uczestników,
\item jednocześnie uruchamia klientów sieci BitTorrent,
\item oczekuje aż każdy otrzyma pełny plik,
\item zatrzymuje klientów,
\item zapamiętuje wyniki,
\item przywraca konfigurację.
\end{enumerate}

\end{document}
