\documentclass[00-praca-magisterska.tex]{subfiles}

\begin{document}

\chapter{Wprowadzenie}

Inżynieria oprogramowania bardzo szeroko omawia tematykę testów, na każdym
kroku podkreślając wagę i nieodłączność tej fazy w procesie tworzenia.
Powstający produkt testuje się, aby zapewnić jak najwyższą jakość dostarczanego
oprogramowania. Pod tym ogólnym stwierdzeniem często rozumie się co innego, w
zależności od rodzaju i przeznaczenia programu. Dlatego też w tym rozdziale
przybliżamy zakres zagadnień, na którym skupimy się w pracy.

\section{Dlaczego testujemy}

Oprogramowanie testuje się, żeby sprawdzić jego poprawność. Twórcy zależy na
tym, żeby zminimalizować ilość swoich błędów. Odbiorca natomiast chciałby
potwierdzenia, że otrzymuje produkt, jakiego oczekiwał. Pod tymi słowami kryje
się za każdym razem inny konkret, któremu się przyglądamy, jednak kilka
ogólnych cech pozostaje niezmiennych.

Pierwszą taką cechą jest weryfikacja mierzalnej wartości. Każdy test powstaje w
celu sprawdzenia czy oczekiwane przewidywania są spełnione. Wymagane jest więc
określenie jakich wartości oczekujemy od poszczególnych czynników. Biorąc za
przykład popularny ostatnio ,,szybki Internet'' łatwo pokazać, że samo nazwanie
czegoś szybkim nie przedstawia żadnej cennej informacji, gdyż mówiąc potocznie,
dostawcy usług zawsze dostarczają najszybszego Internetu, a mimo to z roku na
rok udaje się zapewnić jeszcze szybsze łącza.

Obiektywne sformułowanie przedmiotu testu wymienia nie tylko badaną wielkość,
ale też oczekiwaną wartość. Mówienie o szybkości transferu ma sens, kiedy
padają konkretne liczby, ponieważ wtedy można je porównać z przewidywaniami i
stwierdzić czy dana szybkość nas zadowala.

Może zdarzyć się, że nie taka wartości odniesienia nie istnieje. Przy pierwszym
pomiarze czasu wykonania nowego algorytmu możemy tylko zgadywać lub
subiektywnie oceniać go w świetle podobnych algorytmów. W kolejnych pomiarach
dysponujemy już jednak poprzednią wartością i możemy wnioskować z jej pomocą.
Niezmienny jest jednak fakt, że sprawdzamy wielkość, którą można zmierzyć i
nasze pomiary nie są zaburzone naszą osobistą oceną.

Możliwość dokonania kolejnego pomiaru jest drugą bardzo istotną cechą
testu\footnote{Istnieje szereg testów, których charakter uniemożliwia ich
powtórzenie lub czyni je kosztownym. Sprzęt sieciowy naszczęście nie ulega
zniszczeniu w czasie większości testów.}. Dzięki temu jesteśmy w stanie ocenić
jak na wyniki wpływają takie czynniki jak sprzęt czy zmiany w oprogramowaniu.

Uruchomienie tego samego testu na różnego rodzaju urządzeniach pozwala
stwierdzić czy tworzone rozwiązanie jest przenośne, czy się skaluje, czy
sprawdza się w eksremalnych warunkach itp. W zastosowaniach związanych z
sieciami komputerowymi tego typu testy są szczególnie istotne ze względu na
różnorodność dostępnych urządzeń i protokołów.

Zmiany w oprogramowaniu są naturalne. Zmienia się je tworząc, poprawiając i
rozbudowując. Możliwość wielokrotnego testowania funkcjonalności pozwala ocenić
ją w świetle wprowadzanych zmian. Najpierw można przekonać się o tym, że
pożądana cecha została osiągnięta. Następnie, w przypadku poprawek istniejącego
produktu, możemy przekonać się, że problemy zostały usunięte, a także, że ich
eliminacja nie zaburzyła innych komponentów. To samo dotyczy nowej
funkcjonalności. Mamy też możliwość oceny wpływu zmian na wydajność i to, przy
systematycznym wykonywaniu testów, na przestrzeni życia całego projektu.

Ponieważ w środowiskach rozproszonych powszechna jest sytuacja, w której mamy
do czynienia ze sprzętem różnego rodzaju i różnych producentów, a budowane
rozwiązania mają niezawodnie pracować przez wiele lat, testowanie tworzonych
produktów jest tu bardzo istotne. Jednocześnie jest ono trudniejsze niż w
wypadku rozwiązań przeznaczonych na pojedyncze systemy.

\section{Testy aplikacji sieciowych}

Testowanie sieciowe
\begin{itemize}
  \item{wymienianych PDU,}
  \item{sekwencji komunikatów,}
  \item{wydajność transmisji.}
\end{itemize}

Wyniki. Zapisywać wszystko, dokładnie. Miejsce jest tanie.

\section{Dostępne rozwiązania}

TTCN, ASN1, OMNeT++

\section{Testy w realnych sieciach}

Testy real-life -- czym się różnią od tych w warunkach symulacyjnych.
 - sprzęt
 - wpływ innych użytkowników
 - bezpieczeństwo
 - opóźnienia
 - synchronizacja
 - powtarzalnośc
I nic z tym nie będziemy robić, bo o to chodzi w testach reallife, żeby tak
było.

wymienianych PDU - łatwiej i szybciej jest testować na jednej maszynie, w "sterylnych" warunkach.
wydajność - interesują nas realne, a nie symulowane osiągi

Kwestie bezpieczeństwa.

Oczywiście ponieważ implementacja rozwija się w czasie, a ponadto na pracę
protokołu ma wpływ nie tylko pisany kod, ale również reszta stosu sieciowego,
wszelkie testy najlepiej jest automatyzować, aby można je było wygodnie
powtarzać i dokumentować ich wyniki.

W pracy skupiamy się na protokołach warstwy transportowej i wyższych. Bla bla,
coś na temat tego, że część pojęć (np. PDU) jest bardzo ogólnych i zawężamy ich
znaczenie.

\section{Problematyka automatyzacji testów}

Implementując dowolny protokół szczególną uwagę należy przywiązać do jego
zgodności ze specyfikacją i założeniami dotyczącymi obszaru zastosowań.
W procesie tym pomocne są narzędzia, które sprawdzają poprawność:
\begin{itemize}
  \item{wymienianych PDU,}
  \item{sekwencji komunikatów,}
  \item{wydajność transmisji.}
\end{itemize}

Pierwsze zagadnienie dotyczy budowy komunikatów: długość PDU, położenie i sposób
obliczania sumy kontrolnej, występowanie numeru wersji, itp. Testowanie na tym
poziomie polega na porównywaniu cech obserwowanych PDU z ich specyfikacją. Od
narzędzia wymaga się możliwości sformalizowania opisu zawartego w dokumentacji w
taki sposób, aby mogło samodzielnie analizować komunikaty.

Skuteczna komunikacja wymaga często takich mechanizmów jak nawiązywanie
połączenia czy negocjacja opcji. Wiążą się one z wymianą określonych sekwencji
komunikatów. Wymaga to nie tylko parsowania PDU ale też obserwacji globalnego
stanu połączenia.

Wspomniane do tej pory aspekty, o ile dobierze się odpowiednie narzędzia, nie
przysparzają problemów w czasie testów. Język TTCN-3\footnote{FIXME link?}
został zaprojektowany aby umożliwić analizę implementacji dowolnych protokołów
zarówno w środowiskach wirtualnych jak i rzeczywistych.

Wydajność transmisji dotyczy między innymi osiąganych przepustowości, reakcji na
zatory, jednoczesnych transmisji, rywalizacji z innymi przepływami i innych
czynników mających wpływ na odczuwalną jakość połączenia.

Test w tym przypadku nie polega zatwierdzeniu lub odrzuceniu pojedynczego
przypadku, a na analizie wielokrotnie powtarzanych eksperymentów i odniesieniu
wyników do teoretycznych założeń.

\subsection{Testy wydajności}

Do pomiaru transferów istnieje sporo narzędzi. Problem z pomiarem
wydajności to konfiguracja środowiska, w którym można wykonywać
powtarzalne pomiary. Trzeba tu więc metod opisujących konfiguracje
prostych sieci, przepustowości łączy, procentu gubionych pakietów, itp,
a także sposobu automatycznego zastosowania zdefiniowanej konfiguracji
na zadanych maszynach/routerach.

\end{document}
