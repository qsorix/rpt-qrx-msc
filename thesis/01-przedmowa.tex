\documentclass[00-praca-magisterska.tex]{subfiles}

\begin{document}

\chapter{Wstęp}

W czasie pracy nad jednym z projektów mieliśmy za zadanie sprawdzić wydajność
transmisji ówczesnej implementacji DCCP w kernelu Linuksa. Sam pomiar nie był
uciążliwy i sprowadzał się do uruchomienia programu na dwóch komputerach,
odczekaniu kilkudziesięciu sekund i zapisaniu wyniku.

Kłopot polegał na tym, że trzeba było zrobić to kilkanaście razy przy różnych
parametrach, czasem odwracając role komputerów. Kiedy wprowadziliśmy trzecie
urządzenie, wszystko skomplikowało się jeszcze bardziej.

Czasem test trzeba było powtarzać, bo zapomnieliśmy ustawić odpowiedniej
konfiguracji, czasem nie pamiętaliśmy czego dotyczyły otrzymane wyniki i
uruchamialiśmy test jeszcze raz, tylko po to, żeby się upewnić.

Traciliśmy sporo czasu na robieniu rzeczy, które powinny robić się same. Tak
zrodził się pomysł stworzenia narzędzia, które pomogłoby nam usystematyzować
przeprowadzanie odpowiednich pomiarów.

W pracy tej prezentujemy rozwiązanie, które, osobom wykonującym pomiary na
kilku urządzeniach jednocześnie, pomoże zaoszczędzić wiele godzin traconych na
robieniu małych, prostych czynności. Za każdym razem zajmują one tylko
kilkanaście sekund, ale wyeliminowanie ich pozwala skupić się na tym, co w
testowaniu jest najważniejsze -- na wynikach -- i pracować efektywniej.

\end{document}
