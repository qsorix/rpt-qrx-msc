\documentclass[00-praca-magisterska.tex]{subfiles}

\begin{document}

\chapter{Wprowadzenie}

\section{Geneza}

W czasie pracy nad jednym z projektów w czasie studiów mieliśmy za zadanie
sprawdzić wydajność transmisji ówczesnej implementacji protokołu DCCP w kernelu
Linuksa.  Sam pomiar nie był uciążliwy i sprowadzał się do uruchomienia
programu na dwóch komputerach, odczekaniu kilkudziesięciu sekund i zapisaniu
wyniku.

Kłopot polegał na tym, że pomiary musieliśmy wykonywać po kilka razy dla
każdej z wielu przygotowanych konfiguracji. Często dwie konfiguracje różniły się
wyłącznie tym, który z komputerów biorących udział w teście pełnił rolę klienta,
a który serwera, a wszystko znacznie się komplikowało gdy test wymagał
wykorzystania więcej niż dwóch komputerów. Niekiedy testy należało powtarzać,
ponieważ uruchamialiśmy je korzystając z nieodpowiedniej konfiguracji. Innym
razem nie pamiętaliśmy czego dotyczyły otrzymane wyniki i uruchamialiśmy testy
ponownie, wyłącznie po to, by się o tym przekonać.

Traciliśmy sporo czasu na czynnościach, które powinny być wykonywane
automatycznie. W czasie poszukiwań środowiska, które mogłoby w tym pomóc
natrafiliśmy na kilka istniejących narzędzi, lecz żadne z nich nie pozwalało
na przeprowadzenie testów w interesujący nas sposób. Rozwiązania te zostały
szerzej opisane w rozdziale \fixref{o istniejących rozwiązaniach}. Tak
zrodził się pomysł stworzenia narzędzia, które pomogłoby usystematyzować
przeprowadzanie testów aplikacji sieciowych.

W niniejszej pracy zaprezentujemy rozwiązanie, które pomoże zaoszczędzić wiele
godzin spędzanych na wykonywaniu małych, prostych czynności wymaganych podczas
przeprowadzania pomiarów na kilku urządzeniach sieciowych w tym samym czasie.
Za każdym razem zajmują one tylko kilkanaście sekund, ale wyeliminowanie ich
pozwala pracować efektywniej i skupić się na tym, co w testowaniu
najważniejsze, czyli wynikach.

\section{Cele}

W pracy skupiliśmy się na analizie problemu jakim jest automatyzacja
testowania, przeglądzie istniejących rozwiązań i narzędzi w tym zakresie oraz
omówieniu i ocenie narzędzia Arete (Automated Real Environment Test Engine) --
stworzonego przez nas środowiska do automatyzacji testów.

Przedstawimy problemy związane z testowaniem aplikacji i protokołów sieciowych.
Zademonstrujemy zastosowanie Arete do wykonywania różnego rodzaju testów oraz
wyjaśniamy w jakich przypadkach nie będzie ono odpowiednie. Poruszymy również
kwestie poprawnego przeprowadzania testów oraz katalogowania wyników.

\section{Pojęcia}

W pracy posługujemy się następującymi terminami. 

\begin{description}
  \item[Urządzenie] \hfill \\
Pod tym pojęciem, o ile nie zostanie zaznaczone inaczej, będziemy rozumieć
aktywny sprzęt sieciowy w szczególności wszelkiego rodzaju zarządzalne
przełączniki, rutery oraz komputery.
\end{description}

\section{README}

Część przypisów przeznaczona jest dla promotora. Tak, jak np. ten\promotor{Tekst pisany na niebiesko służy do zaznaczania pytań do promotora.}.

\fixme{Tekst na czerwono to nasze notatki i miejsca, w których trzeba coś poprawić.}

\fixref{W ten sposób oznaczane są brakujące odniesienia do innych fragmentów pracy.}

\end{document}
