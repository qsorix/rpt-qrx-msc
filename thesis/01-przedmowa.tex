\documentclass[00-praca-magisterska.tex]{subfiles}

\begin{document}

\chapter{Przedmowa}

W czasie pracy nad jednym z projektów w czasie studiów mieliśmy za zadanie
sprawdzić wydajność transmisji ówczesnej implementacji DCCP w kernelu Linuksa.
Sam pomiar nie był uciążliwy i sprowadzał się do uruchomienia programu na dwóch
komputerach, odczekaniu kilkudziesięciu sekund i zapisaniu wyniku.

Kłopot polegał na tym, że pomiary musieliśmy wykonywać po kilkanaście razy dla
każdej z przygotowanych konfiguracji. Często dwie konfiguracje różniły się
wyłącznie tym który z komputerów biorących udział w teście pełnił rolę klienta,
a który serwera, a wszystko znacznie komplikowało się gdy test wymagał
wykorzystania więcej niż dwóch komputerów.

Niekiedy testy należało powtarzać, ponieważ uruchamialiśmy je korzystając z
nieodpowiedniej konfiguracji. Innym razem nie pamiętaliśmy czego dotyczyły
otrzymane wyniki i uruchamialiśmy testy ponownie, wyłącznie po to, by się o tym
przekonać.

Traciliśmy sporo czasu na czynnościach, które powinny być wykonywane
automatycznie. Tak zrodził się pomysł stworzenia narzędzia, które
pomogłoby usystematyzować przeprowadzanie testów aplikacji sieciowych.

W miniejszej pracy zaprezentujemy rozwiązanie, które pomoże zaoszczędzić wiele
godzin traconych na wykonywaniu małych, prostych czynności osobom wykonującym
pomiary na kilku urządzeniach\footnote{\emph{Jakich urządzeniach?}}
jednocześnie\footnote{\emph{Co to znaczy jednocześnie?}}. Za każdym razem
zajmują one tylko kilkanaście sekund, ale wyeliminowanie ich pozwala skupić się
na tym, co w testowaniu jest najważniejsze -- na wynikach -- i pracować
efektywniej.

\section{Cele}

\FIXME{ Wymyślić NAZWĘ.  }

W pracy skupiliśmy się na analizie problemu jakim jest automatyzacja
testowania, przeglądzie istniejących rozwiązań i narzędzi w tym zakresie oraz
omówieniu i ocenie narzędzia NAZWA - stworzonego przez nas środowiska do
automatyzacji testów.

Przedstawimy problemy związane z testowaniem aplikacji i protokołów sieciowych.
Zademonstrujemy zastosowanie NAZWA do wykonywania różnego rodzaju testów oraz
wyjaśniamy w jakich przypadkach nie będzie ono odpowiednie.  Poruszymy również
kwestie poprawnego przeprowadzania testów oraz katalogowania wyników.
Testowanie nie jest jednak głównym tematem pracy i informacji oraz wytycznych
należy szukać w innych źródłach.

\section{Plan pracy}

\FIXME{ patrz plan-pracy.txt

Kilka słów o użytych rodzajach \code{formatowania}? \emph{O co chodzi?} }

\end{document}
