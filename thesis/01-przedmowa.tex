\documentclass[00-praca-magisterska.tex]{subfiles}

\begin{document}

\chapter{Przedmowa}

W czasie pracy nad jednym z projektów w czasie studiów mieliśmy za zadanie
sprawdzić wydajność transmisji ówczesnej implementacji DCCP w kernelu Linuksa.
Sam pomiar nie był uciążliwy i sprowadzał się do uruchomienia programu na dwóch
komputerach, odczekaniu kilkudziesięciu sekund i zapisaniu wyniku.

Kłopot polegał na tym, że pomiary musieliśmy wykonywać po kilka razy dla
każdej z wielu przygotowanych konfiguracji. Często dwie konfiguracje różniły się
wyłącznie tym który z komputerów biorących udział w teście pełnił rolę klienta,
a który serwera, a wszystko znacznie komplikowało się gdy test wymagał
wykorzystania więcej niż dwóch komputerów.

Niekiedy testy należało powtarzać, ponieważ uruchamialiśmy je korzystając z
nieodpowiedniej konfiguracji. Innym razem nie pamiętaliśmy czego dotyczyły
otrzymane wyniki i uruchamialiśmy testy ponownie, wyłącznie po to, by się o tym
przekonać.

Traciliśmy sporo czasu na czynnościach, które powinny być wykonywane
automatycznie. Tak zrodził się pomysł stworzenia narzędzia, które
pomogłoby usystematyzować przeprowadzanie testów aplikacji sieciowych.

W miniejszej pracy zaprezentujemy rozwiązanie, które pomoże zaoszczędzić wiele
godzin spędzanych na wykonywaniu małych, prostych czynności wymaganych podczas
przeprowadzania pomiarów na kilku urządzeniach sieciowych w tym samym czasie.
Za każdym razem zajmują one tylko kilkanaście sekund, ale wyeliminowanie ich
pozwala skupić się na tym, co w testowaniu jest najważniejsze -- na wynikach --
i pracować efektywniej.

\section{Cele}

W pracy skupiliśmy się na analizie problemu jakim jest automatyzacja
testowania, przeglądzie istniejących rozwiązań i narzędzi w tym zakresie oraz
omówieniu i ocenie narzędzia Arete (Automated Real Environment Tests Engine) --
stworzonego przez nas środowiska do automatyzacji testów.

Przedstawimy problemy związane z testowaniem aplikacji i protokołów sieciowych.
Zademonstrujemy zastosowanie Arete do wykonywania różnego rodzaju testów oraz
wyjaśniamy w jakich przypadkach nie będzie ono odpowiednie.  Poruszymy również
kwestie poprawnego przeprowadzania testów oraz katalogowania wyników.
Testowanie nie jest jednak głównym tematem pracy i informacji oraz wytycznych
należy szukać w innych źródłach.

\section{Pojęcia i skróty}

W pracy posługujemy się terminem urządzenie. Pod pojęciem tym, o ile nie
zostanie zaznaczone inaczej, będziemy rozumieć aktywny sprzęt sieciowy w
szczególności wszelkiego rodzaju zarządzalne switche, routery oraz komputery.

\section{README}

Część przypisów przeznaczona jest dla promotora. Tak, jak np. ten\promotor{Tekst pisany na niebiesko służy do zaznaczania pytań do promotora.}.

\fixme{Tekst na czerwono to nasze notatki i miejsca, w których trzeba coś poprawić.}

\fixref{W ten sposób oznaczane są brakujące odniesienia do innych fragmentów pracy.}

\section {Skróty}

\FIXME{
Tutaj, jeśli trzeba, możemy zebrać używane skróty. Jak nie trzeba to też możemy
zebrać, będzie pół strony więcej :P
}

\section{Plan pracy}

\FIXME{
patrz plan-pracy.txt
}

\end{document}
