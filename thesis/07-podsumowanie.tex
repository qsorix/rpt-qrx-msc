\documentclass[00-praca-magisterska.tex]{subfiles}

\begin{document}

\chapter{Podsumowanie}

\FIXME{
W tym rozdziale przedstawimy naszą ocenę pracy, tj. na ile jesteśmy zadowoleni
z otrzymanego rozwiązania w świetle na wstępie poczynionych założeń.

Omówimy dalsze możliwości rozwoju, tj. funkcjonalność, której dodanie w
najbliższym czasie będzie użyteczne. Tutaj też zaznaczymy, że warto byłoby
stworzyć plugin frontend współpracujący z routerami cisco.

Inny pomysł rozwoju to możliwość kontrolowania urządzeń będących uruchamianymi
instancjami maszyn wirtualnych (tj. w taki sposób, aby to framework uruchamiał
wirtualne maszyny, a nie zakładał, że już są uruchomione).

Kolejna rzecz to wprowadzenie możliwości wykonywania komend w bardziej złożony
sposób, np. 5 sekund po innej komendzie. Albo co 2 sekundy, ale między 10
sekundą, a zakończeniem pracy innej komendy. To być może będzie wymagać
kompletne innego podejścia do tworzenia planu testu.

Poza tym jakieś podsumowanie całości pracy w paru zdaniach, żeby znudzony
czytelnik przypomniał sobie, o czym tak właściwie pisaliśmy.
}



\end{document}
