\documentclass[00-praca-magisterska.tex]{subfiles}

\begin{document}

\chapter*{Podsumowanie}

W pracy zaprezentowaliśmy stworzone przez nas narzędzie Arete, którego zadaniem jest ułatwienie
automatyzacji testów w środowiskach rozproszonych. Arete jest przydatne zarówno
w czasie przeprowadzania powtarzalnych testów, jak również eksperymentów
mających charakter jednorazowy. Istotą narzędzia jest koordynowanie
uruchamianiem poleceń na wielu urządzeniach oraz gromadzeniem ich wyników.

Pracę rozpoczęliśmy od zbadania oczekiwań, jakie programiści i testerzy wiążą z testowaniem oraz
automatyzacją testów. Pozwoliło nam to sprecyzować wymagania odnośnie narzędzia
oraz sformułować pewne założenia co do obszaru jego zastosowań.  Następnie
omówiliśmy architekturę Master--Slave wybraną do realizacji programu oraz
architektury i sposób realizacji poszczególnych komponentów. Dokładnie
omówiliśmy format konfiguracji używany w celu opisu testu. Użyteczność narzędzia
zademonstrowaliśmy na trzech przykładach realnych zastosowań. W pracy
dokonaliśmy też krytycznej analizy naszego rozwiązania porównując je z
pokrewnymi programami oraz konfrontując je z negatywnymi opiniami o
automatyzacji testów.

Tworząc Arete mieliśmy na uwadze spostrzeżenia z \cite{snake-oil} i
świadomi ograniczeń, wybraliśmy do realizacji pewien zbiór możliwej
funkcjonalności. Dołożyliśmy wszelkich starań, aby zachować pełną możliwość
rozbudowy Arete o dodatkowe możliwości. Mieliśmy na uwadze przede wszystkim
możliwość współpracy z zarządzalnym sprzętem sieciowym ale też z innymi
urządzeniami. Tworzenie dodatkowych interfejsów do komunikacji jest naszym
zdaniem priorytetowym kierunkiem rozwoju narzędzia. W dodatku \ref{reference:interfejs-programowy}
prezentujemy interfejs programowy, który pozwala tworzyć kolejne funkcjonalne
komponenty Arete.

Jesteśmy zadowoleni z osiągniętego rezultatu. Jak zademonstrowaliśmy w
rozdziale \ref{demonstracja-arete} narzędzie Arete jest gotowe do pracy i w
bardzo wielu typach testów jego obecna funkcjonalność będzie w pełni
wystarczająca.  Możemy je polecić studentom i pracownikom akademickim
realizującym zajęcia laboratoryjne w środowiskach rozproszonych.

\end{document}
