\documentclass[00-praca-magisterska.tex]{subfiles}
\begin{document}

\chapter{System Arete}
\label{arete}

Arete (Automated Real Environment Testing Engine) to środowisko automatyzacji
testów, którego stworzenie jest celem naszej pracy. Zadaniem Arete jest
automatyzacja powtarzających się czynności występujących przy przeprowadzaniu
testów w środowiskach rozproszonych w warunkach rzeczywistych.

Analizę rozpoczynamy od przedstawienia, popartych przykładami, zastosowań, z
myślą o których tworzone jest Arete. Są to sytuacje, w których nasze narzędzie
pozwoli usprawnić pracę osób przeprowadzających testy i badania.

Następnie opisujemy wynikające z zastosowań wymagania funkcjonalne. Wymagania
te krótko podsumowują podstawowe możliwości, jakie oferować musi program.

W dalszej części prezentujemy dodatkowe założenia mające wpływ na charakter
tworzonego rozwiązania, jednocześnie argumentując podjęte przez nas decyzje w
świetle jego użyteczności.

Rozdział zamyka prezentacja zaprojektowanej architektury programu.

\section{Potencjalne zastosowania}
\label{arete-zastosowania}

Arete jest tworzone z myślą o osobach testujących, zespołach badawczych oraz
studentach kierunków informatycznych zajmujących się tworzeniem oprogramowania
sieciowego lub przeprowadzaniem eksperymentów w sieciach komputerowych. Może
być ono również przydatne dla prowadzących zajęcia praktyczne, przygotowujących
ćwiczenia w takich środowiskach.

Potrzebne jest narzędzie, które zautomatyzuje najczęściej wykonywane
czynności: skonfiguruje maszyny, przeprowadzi testy, zapamięta wyniki i
przywróci pierwotną konfigurację.

Samo zaprojektowanie testu i stworzenie programów potrzebnych do jego
przeprowadzenia (m.in. programów mierzących obserwowane wartości) pozostanie w
kwestii użytkownika. Test modelowany będzie na poziomie poleceń systemowych,
gdyż pracując ręcznie, użytkownik najczęściej działa na tej płaszczyźnie.

Tworząc Arete skupimy się na kilku przypadkach zastosowań, które pojawiły się w
czasie naszych zajęć laboratoryjnych z różnych przedmiotów. Na uwadze będziemy
mieli trzy konkretne scenariusze omówione poniżej. W dalszej części pracy
posłużą one jako przykłady użycia Arete, aby w ten sposób zweryfikować
działanie systemu.

\subsection{Aplikacja klient-serwer}

Najprostszy scenariusz ma miejsce w czasie tworzenia oprogramowania typu
klient-serwer. Załóżmy, że tworzymy prosty program mający umożliwiać transfer
plików. W teście chcemy sprawdzić czy transfer przebiegł poprawnie, tj. czy
zgadza się z nazwa pliku i czy jego zawartość jest identyczna.

Zadanie narzędzia polegałoby tutaj na:
\begin{enumerate}
\item przesłaniu aktualnej implementacji serwera na wyznaczone urządzenie,
\item uruchomieniu serwera i klienta z odpowiednimi parametrami,
\item zakończeniu pracy serwera,
\item sprawdzeniu nazw i sum kontrolnych plików,
\item zapamiętaniu wyników z obu maszyn.
\end{enumerate}

\subsection{Wpływ transmisji TCP na pracę DCCP}

Drugi scenariusz dotyczy pomiaru szybkości transferu danych przy użyciu
protokołu DCCP (Datagram Congestion Control Protocol) oraz wpływu innych
transmisji na tę szybkość. Sposób pomiaru jest prosty. Wykorzystujemy program
\code{iperf}, który umożliwia uruchomienie serwera i klienta wybranego
protokołu, a przy użyciu reguł \code{iptables} mierzymy ilość pakietów i bajtów
całego ruchu sieciowego.

Aby zautomatyzować ten test, program musi:
\begin{enumerate}
\item korzystając z interfejsu \code{sysctl} skonfigurować parametry DCCP,
\item stworzyć reguły \code{iptables} do mierzenia ruchu,
\item przeprowadzić synchronizację, 
\item rozpocząć test,
\item w wyznaczonych momentach rozpocząć lub zakończyć działanie \code{iperf},
\item zbierać wartości liczników \code{iptables},
\item zapamiętać wszystkie uzyskiwane wyniki,
\item przywrócić opcje DCCP i konfigurację \code{iptables}.
\end{enumerate}

Ważne jest też, żeby w czasie testu program nie przesyłał żadnych danych, aby
nie wpływał w ten sposób na uzyskiwane wyniki.

\subsection{Dystrybucja pliku w sieci BitTorrent}

Trzeci scenariusz, który mamy na uwadze, to ćwiczenie laboratoryjne z Systemów
Peer\--to\--Peer. Test polega na pomiarze czasu dystrybucji pliku w sieci
BitTorrent rozpiętej nad dwiema sieciami IP. Klienci mają rozpocząć pobieranie
w tym samym momencie, a test należy zakończyć kiedy wszyscy będą posiadać pełny
plik. Interesuje nas, ile każdy klient wysłał danych i ile trwała pełna
dystrybucja pliku.

Program w tym teście:
\begin{enumerate}
\item do każdego uczestnika wysyła plik torrent,
\item konfiguruje interfejsy routera i pozostałych maszyn,
\item jednocześnie uruchamia klientów sieci BitTorrent,
\item oczekuje aż każdy otrzyma pełny plik,
\item zatrzymuje klientów,
\item zapamiętuje wyniki,
\item przywraca konfigurację.
\end{enumerate}

Do powyższych scenariuszy wracamy ponownie w rozdziale \ref{demonstracja-arete}. Są
tam one dokładnie omówione, przedstawiony jest też sposób przeprowadzenia ich
przy użyciu Arete.

Warto zauważyć, że w każdym ze wspomnianych scenariuszy test powinien się
zakończyć w innym momencie. W pierwszym jest to moment, kiedy pracę skończy
dany program -- klient. Testy z drugiego scenariusza kończą się w określonym momencie
czasowym. W ostatnim wypadku decyzja o zakończeniu jest podejmowana przez
wszystkich uczestników -- gdy każdy zakończył pobieranie pliku.

\section{Wymagania}
\label{arete-wymagania}

Lista poniżej zbiera wymagania stawiane przed programem we wprowadzeniu oraz te
wynikające z omówionych przed chwilą scenariuszy. W skład możliwości muszą wchodzić:
\begin{itemize}
\item definiowanie testu,
\item łączenie ze zdalnymi urządzeniami,
\item konfiguracja zdalnych urządzeń,
\item przesyłanie zasobów\footnote{np.~pliki konfiguracyjne wykonywanych programów} wymaganych w testach,
\item uruchamianie programów,
\item pobieranie wyników,
\item katalogowanie wyników wielu testów,
\item synchronizacja startu testu,
\item rozłączanie na czas wykonywania testu,
\item zakończenie testu w określony sposób:
  \begin{itemize}
  \item konkretny czas,
  \item wspólna decyzja uczestników testu,
  \item wyznaczony host kończy test.
  \end{itemize}
\end{itemize}

Powyższe wymagania będziemy mieć na uwadze projektując projekt. Będziemy do
nich powracać w dalszej treści pracy, żeby wykazać, iż zostały one spełnione.

\section{Założenia projektowe}
\label{arete-zalozenia}

Realizacja wymienionych w poprzedniej części wymagań wiązała się z podjęciem
wielu decyzji projektowych, które miały istotny wpływ na ogólny obraz tworzonego
programu.  Podejmując je kierowaliśmy się głównie chęcią stworzenia wygodnego i
prostego narzędzia.

\paragraph{Użytkownik wie, co robi.} Przyjęliśmy więc, że nie będziemy
wprowadzać rygorystycznej kontroli poprawności konfigurowanych akcji kosztem
funkcjonalności. Arete stara się ograniczać użytkownika w jak najmniejszym stopniu,
dzięki czemu może być wykorzystane w większej liczbie przypadków.

\paragraph{Bezpieczeństwo jest adekwatne do zagrożeń.} Aby możliwe było
przeprowadzenie pewnych testów, na urządzeniach wymagany jest dostęp w trybie
administratora. Ponieważ polecenia realizujące test przekazywane są zdalnie,
istnieje ryzyko wykorzystania ich do wykonania dowolnych komend z uprawnieniami
administratora. Nie niesie to jednak zagrożeń dla bezpieczeństwa gdyż środowisko
przeprowadzania testu zwykle odpowiada jego wymaganiom dotyczącym uprawnień w
dostępie.

Urządzenia w laboratoriach są przeznaczone do wykonywania na nich
eksperymentów.  Ich pierwotna konfiguracja jest po wykonaniu testów
przywracana. Sieci takie są najczęściej odizolowane, a więc szansa zdalnego
dostępu do tego typu sieci jest znikoma.

Sieci domowe składają się z urządzeń, których administratorem jest właściciel i
najczęściej jest to ta sama osoba, która wykonuje test. Od sieci laboratoryjnych
konfiguracje takie różnią się zwykle faktem połączenia z Internetem i
występowaniem prywatnych danych. Dostęp z zewnątrz zwykle ograniczony jest przez
firewall routera. Użycie naszego programu nie powinno więc zwiększać
istniejącego ryzyka, w przypadku zastosowania go na urządzeniu pełniącym jego
rolę. Jednocześnie mamy pełne zaufanie do użytkowników sieci lokalnej.

Kwestie bezpieczeństwa są istotne w przypadku zastosowania naszego
oprogramowania w sieciach produkcyjnych jak sieć akademicka lub firmowy
intranet. W tego typu sieciach administracją zajmują się wyznaczone osoby.
Użytkownicy nie mają uprzywilejowanego dostępu do urządzeń i zwykle mogą
korzystać tylko z wyznaczonych systemów, na których muszą się uwierzytelniać.
Użytkownikom często nie wolno instalować własnego oprogramowania, a komunikacja
sieciowa jest ograniczana. W takich warunkach korzystanie z naszego programu
jest technicznie ograniczone do czynności, które użytkownik ma prawo wykonywać. 

Zadbaliśmy jednak o to, aby automatyzacja nie wymagała naruszania
bezpieczeństwa poufnych informacji. Zachęcamy użytkowników do korzystania z
mechanizmów autoryzacji przy użyciu kluczy szyfrujących. Nie zawsze jest to
możliwe i wtedy zwykle pozostają zwykłe hasła, zapewnimy jednak mechanizmy
dzięki którym nie trzeba ich wpisywać do konfiguracji, gdzie byłyby narażone na
dostęp przez osoby postronne.

\paragraph{Architektura Master-Slave.} Ponieważ wymagane jest, aby dało się
wykonywać test w sytuacji kiedy nie ma połączenia między urządzeniem, z którego
uruchamiamy test, a innym w teście uczestniczącym, konieczne było wprowadzenie
dodatkowego programu pomocniczego. Wyróżniamy więc program nadzorujący
przebieg testu (master), oraz programy wykonujące go (slave), zainstalowane na
występujących w teście urządzeniach.

Wprowadzenie takiej architektury powoduje, że instalacja programu wykonującego
testy (slave) zajmuje więcej czasu, trzeba ją bowiem przeprowadzić na
wszystkich systemach. Jest to jednak konieczne, w celu osiągnięcia
funkcjonalności wymaganej w pewnych scenariuszach testów, jak również przydatne
ze względów bezpieczeństwa.

W przypadkach, gdy na urządzeniu nie można zainstalować modułu
slave\footnote{Jak np.~większość routerów sprzętowych}, użytkownik ma możliwość
dostarczenia własnej wtyczki, dzięki której wciąż będzie można przeprowadzić
test. Więcej na ten temat w sekcji \ref{arete-master-komunikacja-z-modulem-slave}.

\paragraph{Synchronizacja urządzeń nie zmienia wskazań ich zegarów.}
Najprostszym rozwiązaniem problemu synchronizacji byłoby wykorzystanie ogólnie
dostępnego protokołu synchronizacji czasu (np.~NTP) do ustawienia czasu
systemowego i następnie korzystanie z lokalnych wskazań tych zegarów. Takie
wyjście ma jednak dwie wady. Po pierwsze zmiana zegara nie jest dostępna na
wszystkich systemach. Po drugie, zmiana wskazań zegara w czasie pracy systemu
mogłaby zaburzyć pracę innych uruchomionych programów.

Z tego powodu wprowadzimy protokół synchronizacji, który oblicza różnice między
zegarami urządzeń i uwzględnia je przy wyznaczaniu czasu rozpoczęcia testu.

Protokół ten ma działać w następujący sposób. Moduł master wysyła do modułu
slave zapytanie o lokalny czas urządzenia, na którym jest uruchomiony.
Zakładając symetryczny czas komunikacji w obie strony, różnicę czasu można
wyznaczyć wzorem: $$\Delta t = r - \frac{l_1 + l_2}{2}$$ gdzie $r$ to wskazanie
zegara zdalnego urządzenia w momencie odebrania zapytania o czas a $l_1$ i
$l_2$ to odpowiednio czasy wysłania zapytania i odebrania odpowiedzi. Istnieją
przypadki kiedy czas transmisji w dwóch kierunkach nie jest zbliżony, dzieje
się tak np. kiedy transmisja powrotna odbywa się przy użyciu innego medium lub
gdy wpisy tablic routingu (lub sama topologia sieci) tworzą w obu kierunkach
różne ścieżki. Zastosowanie powyższego wzoru w takich sytuacjach owocuje
większym błędem. Błąd ten jest równy wartości bezwzględnej różnicy czasów
transmisji w obu kierunkach. Warto dodać, że nawet w najprostszych sieciach
Ethernet czas transmisji także jest tylko zbliżony do równego i wciąż mamy do
czynienia z pewnym błędem.

Synchronizacja ta powinna być opcjonalna, ale domyślnie włączona. Jeżeli
przypadek testowy będzie wymagać dokładnej synchronizacji to niestety, z uwagi
na wspomniany problem czasu transmisji, jesteśmy zmuszeni samodzielnie zapewnić
synchronizację urządzeń. W takich testach nasz prosty protokół synchronizacji
mógłby raczej pogorszyć wskazania zegarów\footnote{Przy wsparciu ze strony
sprzętu i systemu operacyjnego protokół NTP osiąga w sieciach 100BASE-T
dokładność rzędu 20 $\mu$s\cite{ntp-research}, nasz algorytm daje błąd mierzony
milisekundami.}, tak więc wskazane może być wyłączenie go. Warto też wziąć pod
uwagę możliwości samego sprzętu, tzn. rozdzielczość zegarów a także
algorytmu planisty w systemie operacyjnym.

\paragraph{Wszystkie wyniki są cenne.} Ponieważ obecnie przechowywanie dużych
ilości danych nie sprawia problemu, w czasie testu staramy się zebrać jak
najwięcej informacji. Ostatecznie to użytkownik decyduje o tym, co zostanie
zapamiętane, ale naszą myślą przewodnią jest umożliwienie zapisania wszystkich
występujących w teście danych, takich jak wyjście programów, wytwarzane pliki,
kody wyjścia uruchamianych poleceń itp.

Ponadto staramy się ułatwić katalogowanie danych, gdyż w przypadku
wielokrotnych uruchomień testu, szybko zbierają się ich duże ilości. Z tego
powodu do ich przechowywania będziemy korzystać z bazy danych SQLite. Jest to
naszym zdaniem odpowiedni kompromis między możliwościami bazy danych a wygodą
przechowywania wyników w zwykłych plikach.

\paragraph{Narzędzie musi być elastyczne.} W czasie pracy zauważyliśmy, że
scenariusze różnego rodzaju testów wymagają różnych podejść.  Zdajemy sobie
jednak sprawę, że stworzenie gotowego narzędzia do przeprowadzania wszystkich
testów nie jest możliwe. Z tego powodu chcemy stworzyć produkt elastyczny,
którego funkcjonalność w wielu miejscach można rozszerzać samodzielnie. Dzięki
temu możliwa będzie obsługa różnego rodzaju urządzeń i topologii sieciowych,
jak również sposobów przeprowadzania testu.

\paragraph{Użytkownicy znają język Python.} Język Python jest wybranym przez nas
językiem implementacji jak i konfiguracji programu. Z tego powodu od
użytkowników wymagana jest jego podstawowa znajomość. Uważamy, że nie jest to
utrudnienie, ponieważ alternatywą byłoby wymaganie, aby użytkownik nauczył się
stosowanej przez nas, nowej składni plików konfiguracyjnych, a znajomość podstaw
języka Python jest bardziej uniwersalna.

Wybór języka Python był podyktowany jego popularnością, mamy więc nadzieję, że
spora część użytkowników odbierze to założenie jako zaletę, a nie wadę
produktu. Kolejną zaletą jest to, że jest on dostępny na wszystkich popularnych
systemach operacyjnych oraz domyślnie zainstalowany na większości systemów
linuksowych.

Poczyniliśmy także założenia mówiące o tym, czego program robić nie będzie.

\paragraph{Analizę wyników pozostawiamy ekspertom.} Program ma pomóc w
pozyskaniu wyników określonych testów. Ich analiza byłaby ona trudna ze
względu na to, że z każdego rodzaju testu wyniki miałyby inny charakter. Poza
tym programy produkują dane w różnych formatach i nie sposób tego wcześniej
przewidzieć. Dostępne są specjalistyczne programy służące do przetwarzania tego
rodzaju danych pomiarowych, uznaliśmy za zbędne i niepoprawne naśladowanie ich
funkcjonalności.

\paragraph{Nie mamy wpływu na fizyczną topologię sieci.} Z oczywistych powodów
program nie ma wpływu na sposób w jaki połączone są urządzenia w laboratorium. W
tworzonych testach zakłada się, że użytkownik wie, jaki wpływ na wyniki testu ma
stworzona przez niego sieć. Program nie stara się analizować zastanej topologii.

\paragraph{Nie wymuszamy konfiguracji środowiska testu.} Przeprowadzając test w
rzeczywistej sieci prawdopodobnie celowo chcemy zbadać zachowanie implementacji
we współpracy z innymi uczestnikami sieci. Nasz program nie wymusza wyłączności
na komunikację sieciową ani nie monitoruje innych użytkowników sieci. Jeśli do
poprawnego przeprowadzenia pomiaru wymagane jest, aby testowany program był
jedynym użytkownikiem sieci, osoba wykonująca test sama musi o to zadbać.

\paragraph{Nie stawiamy wymagań względem urządzeń.} Narzędzie Arete nie
wymaga od urządzeń biorących udział w teście żadnej funkcjonalności.
Dzięki temu jesteśmy w stanie obsługiwać zarówno tanie urządzenia sieciowe, o
bardzo ograniczonych możliwościach konfiguracyjnych, jak i w pełni wyposażone
systemu Uniksowe czy dedykowane routery. To użytkownik decyduje o tym, co w
danym teście mają robić występujące w nim urządzenia i tym samym stawia
im wymagania dotyczące zarówno funkcjonalności, jak i bezpieczeństwa.

\section{Architektura systemu}
\label{arete-architektura}

Rozpoczynając opis architektury powtórzmy jeszcze raz, że narzędzie Arete ma
pracować w sieci komputerowej. Wyróżniamy komputer kontrolujący test, na którym
uruchomiony jest moduł Arete Master, oraz komputery uczestniczące w teście, na
których uruchomiony jest moduł Arete Slave, patrz rysunek \ref{fig:arete-arch}.

\begin{figure}
\begin{center}
\leavevmode
\includegraphics[width=0.6\textwidth]{arete-arch}
\end{center}
\caption{Architektura narzędzia Arete. Moduł kontrolujący -- Master, oraz
moduły wykonawcze -- Slave.}
\label{fig:arete-arch}
\end{figure}

Decyzja o wprowadzeniu takiej architektury wynika wprost z wymagań.
Uruchamianie złożonego planu testu i gromadzenie wyników wymaga programu --
Slave, który te zadania przeprowadzi. Jednocześnie, niezależnie jaki sposób
komunikacji wybierzemy, jeżeli test używa tej samej fizycznej infrastruktury
sieci, może on wymagać, aby nasze narzędzie wstrzymało komunikację, więc
programy muszą być uruchamiane lokalnie na urządzeniach. Wreszcie, niepożądane
jest ręczne uruchamianie testu na kilku urządzeniach, potrzebny jest więc
centralny moduł kontrolujący -- Master.

Chociaż taka architektura wynika z chęci przeprowadzania testu na kilku
urządzeniach, to jednocześnie nie odbiera ona możliwości pracy tylko na jednym.
Możliwa jest zarówno sytuacja, w której moduł Slave zainstalowany jest na
wielu komputerach w sieci, jak i uruchomienie kilku jego instancji na
jednym urządzeniu. Konfiguracja, w której moduły Master i Slave są uruchomione
na tym samym urządzeniu, także jest prawidłowa.

Moduł Slave jest zaprojektowany jako demon pracujący przez cały czas. Master
natomiast uruchamiany jest tylko na czas trwania testu.

Na rysunku \ref{fig:arete-deploy} widać przykładowe rozmieszczenie urządzeń w
sieci i uruchomione na nich moduły systemu. W dalszej części pracy mówiąc o
modułach Master i Slave, jeśli nie precyzujemy, na jakich urządzeniach są
uruchomione, oznacza to, że fizyczna konfiguracja jest dowolna.

\begin{figure}
\begin{center}
\leavevmode
\includegraphics[width=0.7\textwidth]{arete-deploy}
\end{center}
\caption{Diagram wdrożenia narzędzia Arete w sieci komputerowej. Fizyczna
topologia połączeń w sieci (linie przerywane) nie jest istotna, o ile tylko
zapewnia ona poprawną drogę komunikacji między modułami Master i Slave.}
\label{fig:arete-deploy}
\end{figure}

Architektura wewnętrzna modułów Master oraz Slave opisana jest w dwóch
kolejnych rozdziałach. Opis wdrożenia i instalacji narzędzia przedstawiamy w
podrozdziale \ref{instalacja-i-uruchomienie-arete}. Tutaj natomiast przedstawiamy w jaki
sposób moduły te współpracują ze sobą, aby umożliwić przeprowadzenie testu.

Użytkownik rozpoczyna test uruchamiając moduł Master i przekazując mu zadaną
konfigurację testową. Master analizuje otrzymaną konfigurację i przygotowuje
plan działania dla poszczególnych urządzeń, tj. zestaw komend do wykonania oraz
instrukcje o tym, kiedy je wykonać. Przygotowane plany są następnie przesyłane
do modułów Slave i cały system jest gotowy do rozpoczęcia pracy.

W celu spełnienia wymagań stawianych przed programem, wprowadziliśmy cztery
fazy przeprowadzenia testu.

\begin{figure}
\begin{center}
\leavevmode
\includegraphics[width=0.7\textwidth]{arete-test-sequence-deploy}
\end{center}
\caption{Przebieg testu. Faza przygotowywania testu.}
\label{fig:arete-test-seq-deploy}
\end{figure}

Faza pierwsza to sprawdzenie poprawności (rys.~\ref{fig:arete-test-seq-deploy}).
Slave po otrzymaniu konfiguracji sprawdza, czy kontrolowane urządzenie jest w
stanie przeprowadzić test. Faza ta ma na celu eliminację oczywistych błędów,
które wystąpiłyby w czasie testu, a mogą być wcześnie i szybko wykryte.
Przykładem możliwego kryterium poprawności jest dostępność programów dla
wszystkich komend, które mają zostać wykonane.

\begin{figure}
\begin{center}
\leavevmode
\includegraphics[width=0.55\textwidth]{arete-test-sequence-test}
\end{center}
\caption{Przebieg testu. Faza wykonywania testu.}
\label{fig:arete-test-seq-test}
\end{figure}

Wykonanie komend z kolejnych faz testu (rys.~\ref{fig:arete-test-seq-test})
może uniemożliwić dalszą wymianę informacji poprzez sieć, dlatego moduły Master
i Slave mogą zamknąć połączenie po zakończeniu fazy testowej.

Faza druga (oznaczona na rys.~\ref{fig:arete-test-seq-test} kolorem niebieskim)
to konfiguracja urządzenia. Slave na tym etapie wykonuje komendy, które
tymczasowo dostosują środowisko urządzenia do potrzeb testu. Możliwa jest tutaj
np. instalacja reguł zapory ogniowej czy konfiguracja interfejsu sieciowego i
właśnie z tego powodu kanał komunikacyjny między modułami Master i Slave
zostaje wcześniej zamknięty. Dlatego też całość konfiguracji testu przesyłana
jest wcześniej.

Jeżeli na etapie konfiguracji nie wystąpiły żadne błędy, Slave przechodzi do
fazy wykonania testu (kolor czerwony na rys.~\ref{fig:arete-test-seq-test}), tj.
wykonania tych komend, na których faktycznie zależy użytkownikowi. Jest to
główna, zwykle najdłuższa, faza testu.

Faza czwarta (kolor zielony na rys.~\ref{fig:arete-test-seq-test}) to
przywrócenie konfiguracji. Wykonuje się ona zawsze, jeśli tylko test osiągnął
fazę konfiguracji. Komendy zdefiniowane w tej fazie powinny przywrócić
pierwotne środowisko pracy urządzenia, dzięki czemu moduł Master ponownie ma
możliwość nawiązania połączenia i pobrania wyników.

Jak widać, istnieje możliwość, że moduł Master przez długi czas trwania testu
będzie oczekiwał na jego zakończenie bez połączenia z modułem Slave. Jeśli test
ma trwać godzinę, a w pierwszych sekundach (w czasie fazy konfiguracji) wystąpi
błąd, test nie wykona się, ale Master dowie się o tym dopiero po godzinie.
Dlatego właśnie wcześniej występuje faza sprawdzenia poprawności, w czasie
której komunikacja jeszcze jest możliwa.

\begin{figure}
\begin{center}
\leavevmode
\includegraphics[width=0.55\textwidth]{arete-test-sequence-results}
\end{center}
\caption{Przebieg testu. Faza pobierania wyników.}
\label{fig:arete-test-seq-results}
\end{figure}

Po zakończeniu wszystkich faz testu Master zgłasza się do każdego modułu Slave
i pobiera zgromadzone wyniki wszystkich komend
(rys.~\ref{fig:arete-test-seq-results}). Są one zapisywane w lokalnej bazie
danych i stanowią wyjście programu -- wynik testu.

Stwierdzenie, że test się zakończył nie jest zupełnie oczywiste. Zależnie od
konfiguracji, moment ten wyznacza się w inny sposób, o czym pisaliśmy omawiając
wymagania funkcjonalne programu. Dla testów, które mają się zakończyć na
podstawie kooperatywnej decyzji uczestników (jak np. zakończenie testu po
wykonaniu wszystkich komend na jednym z urządzeń) wprowadziliśmy pewien
kompromis w założeniach. Testy takie mogą się wykonywać bez przerywania
połączenia pomiędzy modułami Master i Slave, dzięki czemu Slave może powiadomić
o tym, że zakończył pracę. Na rysunku \ref{fig:arete-test-seq-test} przedstawia
to końcowy komunikat ,,test finished''. Nie jest on przesyłany przy braku
połączenia -- w takich wypadkach test kończy się po upłynięciu zadanego w
konfiguracji czasu.

Problem zakończenia testu był na tyle złożony, że w celu jego rozwiązania
wprowadziliśmy jeszcze jeden mechanizm (wymagający połączeniowego trybu
przeprowadzania testu). Jest to mechanizm powiadomień, który z założenia służyć
miał realizacji scenariusza, w którym test kończy się, kiedy zadana liczba
uczestników wykona wszystkie komendy. Okazało się, że mechanizm jest na tyle
uniwersalny, że można go wykorzystać także do przeprowadzania prostej
synchronizacji w czasie trwania testu.

Powiadomienia (rys.~\ref{fig:arete-test-seq-trigger}) pozwalają stworzyć
licznik (wyzwalacz), który po otrzymaniu określonej liczby powiadomień,
spowoduje wysłanie komunikatu do wszystkich uczestników testu. Wyzwalacze
definiuje się w konfiguracji testu, o której szeroko piszemy w rozdziale
\ref{arete-master}. Licznik jest inicjowany wartością początkową i odlicza w
dół. Komunikat wyzwolenia (aktywacji) wysyłany jest, kiedy osiągnie on zero.
Nie jest narzucone, który Slave może wysyłać powiadomienia. Jeden Slave może
też wysłać ich kilka.

W omawianym przypadku zakończenia testu, wybrane urządzenia wysyłają
powiadomienie gdy zakończą wykonywać swoje komendy. W momencie, kiedy licznik w
module Master osiągnie zero, pozostałe są informowane o tym, aby zakończyły
test. Wystarczy więc ustalić początkową wartość wyzwalacza na ilość urządzeń,
na których zakończenie będziemy oczekiwać.

Do mechanizmu powiadomień powracamy jeszcze prezentując przykładowe
zastosowania programu.

\begin{figure}
\begin{center}
\leavevmode
\includegraphics[width=0.55\textwidth]{arete-test-sequence-trigger}
\end{center}
\caption{Przebieg testu. Mechanizm powiadomień.}
\label{fig:arete-test-seq-trigger}
\end{figure}

\section{Podsumowanie}
\label{arete-podsumowanie}

W rozdziale tym przedstawiliśmy zastosowania, z myślą o których tworzymy
narzędzie Arete. Zastosowania te stawiają szereg wymagań, których spełnienie
było kluczowym zadaniem w celu uzyskania potrzebnej funkcjonalności.

Projektując rozwiązanie, staraliśmy poczynić jak najmniej założeń
wpływających na użytkownika. Kierowaliśmy się przekonaniem, że w ten sposób w
najmniejszym stopniu ograniczymy możliwe spektrum zastosowań i pozostawimy
użytkownikom możliwość adaptacji narzędzia do własnych potrzeb. Zgodnie z
\cite{automation-fail,practical-automated} skupiamy się na wymaganiach
stawianych przez wybrane scenariusze testów i nie próbujemy nadmiernie
komplikować rozwiązania możliwościami dodatkowymi.

Rozdział zakończyliśmy omówieniem architektury master-slave oraz przewidzianego
sposobu działania Arete, dając kompletne wyobrażenie o tym, jakiego rodzaju
jest to narzędzie. W kolejnych rozdziałach pracy szczegółowo omówimy oba moduły
programu.

\end{document}
