\documentclass[a4paper,oneside,12pt]{mwbk}

\usepackage{setspace}
\usepackage[utf8x]{inputenc}
\usepackage{polski}
\usepackage{amsmath}
\usepackage{amsfonts}
\usepackage{url}
%\usepackage[pdfborder={0 0 0}]{hyperref}
%\usepackage{hypcap}
\usepackage{sphinx}
\usepackage{ucs}

\usepackage{subfiles}

\usepackage{color}
\usepackage{graphicx}
\DeclareGraphicsExtensions{.eps,.png,.jpg}
\graphicspath{{./diagrams/}}
\usepackage{array}
\newcolumntype{C}[1]{>{\centering\arraybackslash}p{#1}}

% automatic conversion from .eps to .pdf so pdflatex can be used
\usepackage{epstopdf}

\definecolor{codebg}{rgb}{0.95,0.95,0.95}
\usepackage{minted}
\newminted{python}{gobble=2,fontsize=\small,baselinestretch=1.0,bgcolor=codebg}
\newminted{text}{gobble=2,fontsize=\small,baselinestretch=1.0,bgcolor=codebg}
\newminted{xml}{gobble=2,fontsize=\small,baselinestretch=1.0,bgcolor=codebg}

\usepackage{fullpage}
\usepackage{fancyhdr}
\pagestyle{fancy}
\setlength{\headheight}{15.2pt}
\setlength{\headsep}{15pt}
\setlength{\topmargin}{0pt}
\setlength{\voffset}{-20pt}
\setlength{\footskip}{40pt}
\renewcommand{\chaptermark}[1]{\markboth{\thechapter. #1}{}} % nie robi wielkich liter w naglowkach
\lhead{}
\cfoot{}
\rfoot{\thepage}
\oddsidemargin=0.5cm

%\newcommand{\code}[1] {{\textbf{\texttt{#1}}}}
\newcommand{\FIXME}[1] {
  \vspace{0.2cm}
  \noindent\ignorespaces\fcolorbox{black}{red}{
    \begin{minipage}{1.0\textwidth}
    FIXME

    #1
    \end{minipage}}
  \vspace{0.2cm}}
\newcommand{\fixme}[1] {\textcolor{red}{#1}}

\newcommand{\fixref}[1] {\textit{#1}}

\newcommand{\promotor}[1]{\footnote{\textcolor{blue}{#1}}}
\newcommand{\ang}[1]{ang.~\textit{#1}}

\setlength{\parindent}{0pt}
\setlength{\parskip}{1ex}
\onehalfspacing

\title{Praca Magisterska}
\author{Krzysztof Rutka \and Tomasz Rydzyński}

\begin{document}
\frontmatter

\begin{titlepage}

\begin{center}

\textsc{\LARGE Akademia Górniczo-Hutnicza\\ \Large im. Stanisława Staszica w Krakowie}\\[0.5cm]

Wydział Elektrotechniki, Automatyki, Informatyki i Elektroniki\\
Katedra Informatyki\\
Kierunek: Informatyka\\[1.5cm]

\includegraphics[width=0.22\textwidth]{agh-logo}\\[1.0cm]    

\textsc{\Large praca magisterska}\\[0.5cm]


% Title
\textsc{\LARGE Automatyzacja testów protokołów komunikacji aplikacji sieciowych}\\[0.4cm]

\vfill

% Author and supervisor
\begin{minipage}[t]{0.49\textwidth}
\begin{flushleft} \large
\emph{Autorzy:}\\
Krzysztof \textsc{Rutka} \\
Tomasz \textsc{Rydzyński} \\
\end{flushleft}
\end{minipage}
\begin{minipage}[t]{0.49\textwidth}
\begin{flushright} \large
\emph{Opiekun pracy:} \\
Dr~inż.~Łukasz \textsc{Czekierda}
\vfill
\end{flushright}
\end{minipage}

\vfill

% Bottom of the page
{\large \today}

\end{center}

\end{titlepage}

\clearpage
\thispagestyle{empty}
{\textbf{\Large{Oświadczenie}}}

Oświadczamy, świadomi odpowiedzialności karnej za poświadczenie nieprawdy, że
niniejszą pracę dyplomową wykonaliśmy osobiście i samodzielnie (w zakresie
wyszczególnionym we wstępie) i że nie korzystaliśmy ze źródeł innych niż
wymienione w pracy.
\clearpage

\setcounter{tocdepth}{1}

\begin{spacing}{0.9}
\tableofcontents
\end{spacing}

\subfile{01-przedmowa}

\mainmatter

\subfile{02-testowanie}
\subfile{02-automatyzacja}
\subfile{03-arete}
\subfile{04-arete-master}
\subfile{05-arete-slave}
\subfile{06-przyklady}
\subfile{06-weryfikacja}
\subfile{07-podsumowanie}

\subfile{b-bibliografia}

\appendix
\subfile{a1-skroty}
\subfile{a2-reference}


\end{document}
