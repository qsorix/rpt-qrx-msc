\documentclass[00-praca-magisterska.tex]{subfiles}

\begin{document}

\begin{thebibliography}{9}

% Testowanie (ogólnie):

% Testowanie sieciowe:

% Inżynieria Oprogramowania:

% Sieci:

% DCCP:

% ASN.1+ECN:
\bibitem{asn1-main}
  \emph{ITU-T: Introduction to ASN.1},
  \url{http://www.itu.int/ITU-T/asn1/introduction/index.htm},
  International Telecommunication Union - Telecommunication Standardization Sector

\bibitem{ecn-tutorial1}
  Markku Turunen,
  \emph{A short introduction to ASN.1 Encoding Control Notation (ECN)},
  \url{http://www.itu.int/itu-t/asn1/ecn/ECN_Introduction_June_2001.pdf},
  2001

\bibitem{ecn-tutorial2}
  John Larmouth,
  \emph{ETSI ECN Tutorial},
  \url{http://www.itu.int/itu-t/asn1/ecn/ECN_Tutorial_V1.1_Jan_2001.pdf},
  wersja 1.1,
  2001

\bibitem{ecn-tutorial3}
  John Larmouth,
  \emph{Encoding Control Notation (ECN)},
  \url{http://www.itu.int/itu-t/asn1/ecn/ECN_Tutorial_V1_Dec_2000.pdf},
  wersja 1, 
  2000

% TTCN-3:
\bibitem{ttcn-main}
  \emph{TTCN-3 Standard Suite},
  \url{http://www.ttcn-3.org/StandardSuite.htm}

\bibitem{ttcn-reference}
  \emph{TTCN-3 language reference},
  \url{http://wiki.openttcn.com/media/index.php/OpenTTCN/Language_reference},
  2008

% OMNeT++:
\bibitem{omnet-main}
  \emph{OMNeT++: Network Simulation Framework},
  \url{http://www.omnetpp.org/}

\bibitem{omnet-doc}
  \emph{OMNeT++ documentation and tutorials},
  \url{http://omnetpp.org/documentation},
  wersja 4.1

% Python:
\bibitem{python-doc}
  \emph{Python documentation}, 
  \url{http://docs.python.org/},
  wersja 2.7

\bibitem{python-pep-8}
  Guido van Rossum, Barry Warsaw,
  \emph{Style Guide for Python Code},
  \url{http://www.python.org/dev/peps/pep-0008/}

\bibitem{python-pep-257}
  David Goodger, Guido van Rossum,
  \emph{Python Docstring Conventions},
  \url{http://www.python.org/dev/peps/pep-0257/}

\bibitem{python-pep-263}
  Marc-André Lemburg, Martin von Löwis,
  \emph{Defining Python Source Code Encodings},
  \url{http://www.python.org/dev/peps/pep-0263/}

% Zależności:
\bibitem{elixir}
  \emph{Elixir documentation},
  \url{http://elixir.ematia.de/trac/wiki/Documentation},
  wersja 0.7

\bibitem{sqlalchemy}
  \emph{SQLAlchemy documentation},
  \url{http://www.sqlalchemy.org/docs/},
  wersja 0.6.6

\bibitem{paramiko}
  \emph{paramiko API documentation},
  \url{http://www.lag.net/paramiko/docs/},
  wersja 1.7.4

% Inne narzędzie:
\bibitem{wikibooks-latex}
  \emph{Wikibooks: \LaTeX},
  \url{http://upload.wikimedia.org/wikipedia/commons/2/2d/LaTeX.pdf}

\bibitem{thesis-latex}
  Lapo F. Mori, 
  \emph{Writing a thesis with \LaTeX},
  \url{http://www.scribd.com/doc/2633955/Writing-a-thesis-with-LaTeX},
  2007

\end{thebibliography}

\FIXME{Co jeszcze?
Wszystko na tematy: Testowanie (ogólnie), Testowanie sieciowe, Inżynieria Oprogramowania, Sieci... co mogłoby pasować.

Jakieś RFC o DCCP itd. Może ta magisterka o DCCP?

Jakaś cegła o sieciach (Model OSI, PDU), jakaś cegła o IO

Jakieś jeszcze tutoriale stąd: http://www.ttcn-3.org/CoursesAndTutorials.htm ?

Jakiś konkretny tutorial o OMNecie? http://omnetpp.org/documentation ?
}

\FIXME{Takie to chyba do przypisów jednak:

Wzorzec Visitor: http://www.ishiboo.com/~danny/c++/design\_patterns/hires/pat5kfso.htm
}

\end{document}
