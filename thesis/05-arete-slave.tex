\documentclass[00-praca-magisterska.tex]{subfiles}

\begin{document}

\chapter{Moduł Arete Slave}
\label{arete-slave}

Arete Slave jest programem służącym do wykonywania testów na komputerach
stacjonarnych. Działa on jako
demon\footnote{http://pl.wikipedia.org/wiki/Demon\_(informatyka)} sterowany
przez moduł Master za pomocą protokołu opisanego w rozdziale
\ref{arete-slave-protokol}.

\section{Architektura}
\label{arete-slave-arch}

W skład architektury programu Arete Slave wchodzą cztery moduły oraz baza danych
służąca do przechowywania informacji o uruchamianych testach, jak zostało to
przedstawione na rysunku \ref{fig:arete-slave-arch}.

\begin{figure}[htb]
\begin{center}
\leavevmode
\includegraphics[width=0.8\textwidth]{arete-slave-arch}
\end{center}
\caption{Architektura}
\label{fig:arete-slave-arch}
\end{figure}

\paragraph{Handler} jest modułem obsługującym przychodzące połączenia TCP,
służącym do komunikacji z Arete Master. Korzysta z modułu Parser, aby sprawdzić czy
przychodzące komunikaty są zgodne z protokołem komunikacji (Rozdział
\ref{arete-slave-protokol}). Otrzymane dane przekazuje do modułu Manager, który
zajmuje się ich interpretacją.

\paragraph{Parser} służy do analizy składniowej przychodzących komunikatów.

\paragraph{Manager} to główny moduł zarządzający testami. Na podstawie danych
otrzymanych od modułu Handler tworzy testy, dodaje do nich komendy oraz zapisuje te
informacje do bazy danych. Następnie zajmuje się uruchamianiem i zatrzymywaniem
testów oraz zwracaniem wyników. Komendy główne które mają zostać wykonane w
określonym momencie testu przekazywane są do modułu Scheduler. 

\paragraph{Scheduler} pozwala uruchomić komendy w określonych punktach czasowych
testu. Moduł ten korzysta z bazy danych w celu pobrania niezbędnych danych i
zapisania wyników uruchamianych komend.

\section{Protokół komunikacji z Arete Master}
\label{arete-slave-protokol}

Do komunikacji między modułami Master i Slave zastosowaliśmy prosty protokół
opisany poniżej. Wymiana komunikatów składa się z trzech głównych faz. Na
początku tworzony jest test i odbierane są komendy które zostaną uruchomione.
Następnie wywoływane są polecenia zarządzające wykonaniem testu.  Na koniec
wyniki przeprowadzonego testu są wysyłane do modułu Master. 

Do przesyłania poleceń wykorzystaliśmy następujący format:

\code{<polecenie> @\{<parametr>=<wartość>\} <komenda>}

Polecenie określa czynność która ma zostać wykonana, taką jak stworzenie testu,
jego uruchomienie czy pobranie wyników. Zastosowany format przekazywania
parametrów \code{@\{...\}} pozwala w łatwy sposób wyróżnić je w otrzymanej
wiadomości. W przypadku poleceń \code{check}, \code{setup}, \code{task} i
\code{clean} do testu dodawana jest komenda wraz z argumentami podawana na końcu
wiadomości.

\subsection{Odbieranie planu testu}

Do stworzenia testu oraz odebrania komend wchodzących w jego skład wykorzystane
są następujące polecenia:

\begin{itemize}
  \setlength{\itemsep}{10pt}

\item{\code{test @\{id=<identyfikator>\}} - tworzy nowy test o podanym
identyfikatorze. Odebranie tego polecenia powoduje przejście w tryb testu, w
którym możemy dodawać komendy jednego z czterech typów.}

\item{\code{check @\{id=<identyfikator>\} <komenda>} - dodaje do testu komendę
sprawdzającą o podanym identyfikatorze. Komendy te wywoływane są korzystając z
polecenia \code{prepare}.}

\item{\code{setup @\{id=<identyfikator>\} <komenda>} - dodaje do testu komendę
konfiguracyjną o podanym identyfikatorze. Komendy te uruchamiane są po odebraniu
polecenia \code{start}.}

\item{\code{task @\{id=<identyfikator>\} @\{run=<tryb uruchomienia>\} <komenda>} -
dodaje do testu komendę główną o podanym identyfikatorze uruchamianą w czasie
wykonywania testu zgodnie z parametrem określającym tryp uruchomienia. \\

Dostępne są trzy tryby uruchamiania: \\

\begin{itemize}
  \setlength{\itemsep}{10pt}

\item{\code{at <sekunda>} - uruchamia komendę w podanej sekundzie testu licząc
od daty rozpoczęcia testu.}
\item{\code{after <identyfikator>} - uruchamia komendę po wykonaniu komendy o
podanym identyfikatorze.} 
\item{\code{every <sekund>} - uruchamia komendę cyklicznie co podaną liczbę
sekund licząc od daty rozpoczęcia testu.}

\end{itemize}}

\item{\code{clean @\{id=<identyfikator>\} <komenda>} - dodaje do testu komendę
czyszczącą o podanym identyfikatorze. Komendy te uruchamiane są po zakończeniu
test.}

\item{\code{file @\{id=<identyfikator>\} @\{size=<rozmiar pliku>\}} - dodaje do testu
plik o podanym identyfikatorze i rozmiarze. Pliki przechowywane są w katalogu
\code{tmp} i mogą być wykorzystywane jako parametry komend wykonywanych podczas
testu.}

\item{\code{end} - powoduje wyjście z trybu testu.}

\end{itemize}

\subsection{Zarządzenie testem}

Po poprawnym stworzeniu testu można nim zarządzać przy pomocy
następujących komend:

\begin{itemize}
  \setlength{\itemsep}{10pt}

\item{\code{prepare @\{id=<identyfikator>\}} - uruchamia komendy sprawdzające
testu o podanym identyfikatorze przesłane poleceniem \code{check}. Pozwala to
stwierdzić czy środowisko uruchomienia testu zawiera wszystkie niezbędne do jego
wykonania programy i ustawienia. Umożliwia również zapamiętanie podstawowych
informacji o systemie.  Jeśli któraś z komend nie wykona się poprawnie zostanie
zwrócony komunikat \code{401 Command Failed}.}

\item{\code{start @\{id=<identyfikator>\} @\{run=at <czas>\} @\{end=<tryb
zakończenia>\}} - uruchamia właściwy test o podanym identyfikatorze. Najpierw
wykonywane są komendy konfiguracyjne przesłane poleceniem \code{setup},
a następnie uruchamiany jest scheduler wykonujący komendy główne w
określonych momentach zaczynając od czasu podanego w parametrze \code{run}. Na
koniec wykonywane są komendy czyszczące. Są one również uruchamiane w przypadku
gdy któraś z komend głównych zakończy się błędem. \\

Do dyspozycji są dwa tryby zakończenia testu: \\

\begin{itemize}
  \setlength{\itemsep}{10pt}

\item{\code{duration <sekundy>} - test kończy działanie po określonej liczbie
sekund.}

\item{\code{complete} - test kończy działanie po zakończeniu wykonywania się
ostatniej komendy. Do Arete Master wysyłany jest wtedy komunikat \code{100 Test
Finished}.}

\end{itemize}}

\item{\code{stop @\{id=<identyfikator>\}} - zatrzymuje wykonywanie testu o
podanym identyfikatorze. Wykonywane są również komendy czyszczące wchodzą w
skład zatrzymanego testu. Zachowane zostają parametry tylko tych komend, które
zakończyły się przez wydaniem polecenia \code{stop}.}

\item{\code{delete @\{id=<identyfikator\}} - usuwa test o podanym
identyfikatorze oraz wszystkie należące do niego komendy i pliki.}

\end{itemize}

\subsection{Wysyłanie wyników testu}

Po zakończeniu testu możliwe jest obebranie jest wyników w celu zapisania do
bazy danych modułu Arete Master i dalszej analizy. Wykorzystujemy do tego celu
polecenia:

\begin{itemize}
  \setlength{\itemsep}{10pt}

\item{\code{results @\{id=<identyfikator>\}} - przechodzi w tryb pobierania
wyników testu o podanym identyfikatorze. Jeśli test nie został jeszcze
rozpoczęty lub nie zakończył się zostanie przesłany komunikat o błędzie.}

\item{\code{get @\{<parametr>\}} - zwraca określony parametr testu. \\

Dostępne parametry testu to: \\

\begin{itemize}
  \setlength{\itemsep}{10pt}

\item{\code{checks, setups, tasks, cleans} - listy komend danego typu które
zostały poprawnie uruchomione i posiadają niepuste wyjście.}

\item{\code{start\_time} - czas rozpoczęcia testu}

\item{\code{duration} - długość trwania testu}

\end{itemize}}

\item{\code{get @\{<identyfikator>.<parametr>\}} - zwraca określony parametr
komendy o podanym identyfikatorze. \\

Dostępne parametry komend to: \\

\begin{itemize}
  \setlength{\itemsep}{10pt}

\item{\code{output} - standardowe wyjście}

\item{\code{returncode} - zwrócony kod wyjścia}

\item{\code{start\_time} - czas rozpoczęcia}

\item{\code{duration} - długość trwania}

\end{itemize}}

\item{\code{end} - powoduje wyjście z trybu wyników}

\end{itemize}

\subsection{Podstawianie parametrów}
\label{arete-slave-podst-param}

Komendy przesyłane do modułu Slave w ramach testu mogą korzystać z parametrów
innych komend wchodzących w skład tego samego testu, lecz jest to możliwe
wyłącznie jeśli komendy te zostały wcześniej wywołane. W przeciwnym wypadku
Arete Slave zwróci błąd. Parametry które mogą zostać podstawione to:

\begin{itemize}
  \setlength{\itemsep}{10pt}

\item{\code{@\{<identyfikator>.returncode\}} - kod wyjścia zwrócony przez komendę
o podanym identyfikatorze. Jeśli komenda była wykonywana więcej niż jeden raz
zwracany jest kod wyjścia ostatniego wykonania.}

\item{\code{@\{<identyfikator>.pid\}} - zwraca identyfikator procesu stworzonego
poprzez wywołanie komendy o danym identyfikatorze. Odnosi się wyłącznie w
przypadku komend typu \emph{task}.}

\end{itemize}

Podstawianie parametrów pozwala również na wykorzystanie plików dodanych do
testu. Służą do tego parametry:

\begin{itemize}
  \setlength{\itemsep}{10pt}

\item{\code{@\{<identyfikator>.size\}} - zwraca rozmiar pliku o podanym
identyfikatorze.}

\item{\code{@\{<identyfikator>.path\}} - zwraca pełną ścieżkę dostępu do pliku o
podanym identyfikatorze.}

\end{itemize}

\subsection{Komunikaty}
\label{arete-slave-komunikaty}

Podczas komunikacji Arete Slave wysyła następujące komunikaty, aby
sygnalizować o poprawnym lub błędnym działaniu:

\begin{itemize}
  \setlength{\itemsep}{10pt}

\item{\code{100 Test Finished} - zwracany po wykonaniu wszystkich komend
wchodzących w skład testu. Obowiązuje wyłącznie w trybie \emph{complete}.}

\item{\code{200 OK <lista rozmiarów>} - zwracany do potwierdzenia poprawnie wykonanego
polecenia; w przypadku pobierania wyników wykonanego testu przesyła również
listę rozmiarów kolejnych wyników (np. \code{get @\{<id>.output\}}).}

\item{\code{201 List <lista identyfikatorów>} - zwracany przy zapytaniach o listy
komend które zakończyły się poprawnie (np. \code{get @\{checks\}}).}

\item{\code{400 Bad Request} - zwaracany w przypadku jakiegokolwiek błędu. Najczęściej
świadczy o przesłaniu komunikatu niezgodnego z protokołem lub wykorzystaniu
identyfikatora który już istnieje w bazie danych.}

\item{\code{401 Command Failed <identyfikator>} - zwracany gdy któraś z wykonywanych
komend zakończy się błędem. W celu szybkiego zidentyfikowania problemu
przesyłany jest identyfikator błędnej komendy.}

\item{\code{402 Setup Too Long} - zwracany gdy komendy konfiguracyjne nie zakończyły
się przed planowany czasem rozpoczęcia właściwego testu. Pozwala to zwrócić
użytkownikowi uwagę na potrzebę zwiększenia wartości opóźnienia po którym
należy uruchomić test.}

\end{itemize}

\section{Przechowywanie danych}
\label{arete-slave-baza-danych}

Podstawową jednostką w bazie danych jest test (tabela \emph{tests}) w skład
którego wchodzą komendy (tabela \emph{commands}) i pliki (tabela \emph{files}).
Dodatkowe cztery tabele dziedziczące z tabeli \emph{commands} określają komendy
o konkretnych zastosowaniach (tabele \emph{check\_commands},
\emph{setup\_commands}, \emph{tasks} i \emph{clean\_commands}).

Każda komenda posiada cztery parametry, które są zwracane jako wyniki po
zakończeniu testu: wyjście, zwracany kod wyjścia, czas rozpoczęcia oraz długość
trwania, znajdujące się kolejno w tabelach \emph{output}, \emph{returncode},
\emph{start\_time} oraz \emph{duration}.  Stworzenie osobnych tabel dla tych
wartości umożliwia wielokrotne uruchomienie danej komendy i każdorazowe
zapamiętanie wszystkich parametrów.

\begin{figure}[htb]
\begin{center}
\leavevmode
\includegraphics[width=1\textwidth]{arete-slave-db}
\end{center}
\caption{Schemat bazy danych}
\label{fig:arete-slave-db}
\end{figure}

\section{Uruchamianie}
\label{arete-slave-uruchamianie}

\begin{minted}{text}
  usage: Main.py [-h] [-p PORT] [-d DATABASE] [-l LOG]

  optional arguments:
    -h, --help            show this help message and exit
    -p PORT, --port PORT
                          use port other than 4567
    -d DATABASE, --database DATABASE
                          use database file other than 'aretes.db'
    -l LOG, --log LOG
                          use log file other than 'aretes.log'
\end{minted}

\end{document}
