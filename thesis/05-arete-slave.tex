\documentclass[00-praca-magisterska.tex]{subfiles}

\begin{document}

\chapter{Moduł Arete Slave}

Arete Slave jest programem służącym do wykonywania testów na komputerach
\fixme{nie wiem jak to napisać}.

\section{Architektura}

W skład architektury programu Arete Slave wchodzi pięć modułów oraz baza danych
służąca do przechowywania informacji o uruchamianych testach.

\FIXME{
Opis modułów:

Daemon - konfiguracja bazy danych, server TCP

Handler - obsługa przychodzących połączeń, komunikacja z Arete Master

Parser - parsowanie przychodzących komunikatów

Manager - zarządzanie konfiguracją, dodawanie wpisów do bazy, uruchamianie
testów, zapisywanie wyników...

Scheduler - uruchamianie komend wchodzących w skład testu w odpowiednim momencie
}

\subsection{Przechowywanie danych}

\FIXME{Schemat i opis bazy danych.}

\section{Protokół komunikacji z Arete Master}

\FIXME{Czy można to nazwać protokołem?}

Do komunikacji między modułami Master i Slave zastosowaliśmy prosty protokół
opisany poniżej. Wymiana komunikatów składa się z trzech głównych faz. Na
początku tworzony jest test i odbierane są komendy które zostaną uruchomione.
Następnie wywoływane są polecenia zarządzające wykonaniem testu.  Na koniec
wyniki przeprowadzonego testu są wysyłane do modułu Master. 

Do przesyłania poleceń wykorzystaliśmy następujący format:

\code{<polecenie> @\{<parametr>=<wartość>\} <komenda>}

\fixme{Opisać go...}

\subsection{Odbieranie planu testu}

Do stworzenia testu oraz odebrania komend wchodzących w jego skład wykorzystane
są następujące polecenia:

\code{test @\{id=<identyfikator>\}} - tworzy nowy test o podanym
identyfikatorze. Odebranie tego polecenia powoduje przejście w tryb testu, w
którym możemy dodawać oraz usuwać komendy jednego z czterech typów.

\code{check @\{id=<identyfikator>\} <komenda>} - dodaje do testu komendę
sprawdzającą o podanym identyfikatorze. Komendy te wywoływane są przez testem
korzystając z polecenia \code{prepare}.

\code{setup @\{id=<identyfikator>\} <komenda>} - dodaje do testu komendę
konfiguracyjną o podanym identyfikatorze. Komendy te uruchamiane są po odebraniu
polecenia \code{setup}.

\code{task @\{id=<identyfikator>\} @\{run=<tryb uruchomienia>\} <komenda>} -
dodaje do testu komendę o podanym identyfikatorze uruchamianą w czasie
wykonywania testu zgodnie z parametrem określającym tryp uruchomienia. Dostępne
są 3 tryby uruchamiania.

\begin{itemize} \item{\code{at <sekunda>} - uruchamia komendę w podanej
sekundzie testu licząc od daty rozpoczęcia testu.} \item{\code{after
<identyfikator>} - uruchamia komendę po wykonaniu komendy o podanym
identyfikatorze.} \item{\code{every <sekund>} - uruchamia komendę cyklicznie co
podaną liczbę sekund licząc od daty rozpoczęcia testu.} \end{itemize}

\code{clean @\{id=<identyfikator>\} <komenda>} - dodaje do testu komendę
czyszczącą o podanym identyfikatorze. Komendy te uruchamiane są po zakończeniu
test.

\code{file @\{id=<identyfikator>\} @\{size=<rozmiar pliku>\}} - dodaje do testu
plik o podanym identyfikatorze i rozmiarze. Pliki przechowywane są w katalogu
\code{tmp} i mogą być wykorzystywane jako parametry komend wykonywanych podczas
testu.

\code{end} - powoduje wyjście z trybu testu.

\subsubsection{Podstawianie parametrów}

Komendy przesyłane do modułu Slave w ramach testu mogą korzystać z parametrów
innych komend wchodzących w skład tego samego testu.

\code{@\{<identyfikator>.returncode\}} 

\code{@\{<identyfikator>.pid\}} - tylko task

\code{@\{<identyfikator>.size\}} - tylko plik

\code{@\{<identyfikator>.path\}} - tylko plik

\subsubsection{Komunikaty}

\code{100 Test Finished} - opis

\code{200 OK <lista rozmiarów>} - opis

\code{201 List <lista identyfikatorów>} - opis

\code{400 Bad Request} - opis

\code{401 Command Failed <identyfikator>} - opis

\code{402 Setup Too Long} - opis

\subsection{Zarządzenie testem}

\code{prepare @\{id=<identyfikator>\}} - opis

\code{start @\{id=<identyfikator>\} @\{run=<tryb uruchomienia>\} @\{end=<tryb
zakończenia>\}} - opis

\code{stop @\{id=<identyfikator>\}} - opis

\code{delete @\{id=<identyfikator\}} - opis

\subsection{Wysyłanie wyników testu}

\code{results @\{id=<identyfikator>\}} - opis

\code{get @\{<parametr>\}} - opis

Parametry:

\begin{itemize}
\item{\code{checks, setups, tasks, cleans} - opis}
\item{\code{start\_time} - opis}
\item{\code{duration} - opis}
\end{itemize}

\code{get @\{<identyfikator>.<parametr>\}} - opis

Parametry:

\begin{itemize}
\item{\code{output} - opis}
\item{\code{returncode} - opis}
\item{\code{start\_time} - opis}
\item{\code{duration} - opis}
\end{itemize}

\code{end} - powoduje wyjście z trybu wyników

\end{document}
