\documentclass[a4paper,11pt]{mwart}

\usepackage[utf8]{inputenc}
\usepackage[polish]{babel}
\usepackage{polski}
\usepackage{graphicx}
%\usepackage{color}
%\usepackage{float}
%\usepackage{hyperref}

\begin{document}

\section{Wstęp}

Oczywiście ponieważ implementacja rozwija się w czasie, a ponadto na pracę
protokołu ma wpływ nie tylko pisany kod, ale również reszta stosu sieciowego,
wszelkie testy najlepiej jest automatyzować, aby można je było wygodnie
powtarzać i dokumentować ich wyniki.

W pracy skupiamy się na protokołach warstwy transportowej i wyższych. Bla bla,
coś na temat tego, że część pojęć (np. PDU) jest bardzo ogólnych i zawężamy ich
znaczenie.

\section{Skróty}

PDU - PDU

\section{Problematyka automatyzacji testów}

Implementując dowolny protokół szczególną uwagę należy przywiązać do jego
zgodności ze specyfikacją i założeniami dotyczącymi obszaru zastosowań.
W procesie tym pomocne są narzędzia, które sprawdzają poprawność:
\begin{itemize}
  \item{wymienianych PDU,}
  \item{sekwencji komunikatów,}
  \item{wydajność transmisji.}
\end{itemize}

Pierwsze zagadnienie dotyczy budowy komunikatów: długość PDU, położenie i sposób
obliczania sumy kontrolnej, występowanie numeru wersji, itp. Testowanie na tym
poziomie polega na porównywaniu cech obserwowanych PDU z ich specyfikacją. Od
narzędzia wymaga się możliwości sformalizowania opisu zawartego w dokumentacji w
taki sposób, aby mogło samodzielnie analizować komunikaty.

Skuteczna komunikacja wymaga często takich mechanizmów jak nawiązywanie
połączenia czy negocjacja opcji. Wiążą się one z wymianą określonych sekwencji
komunikatów. Wymaga to nie tylko parsowania PDU ale też obserwacji globalnego
stanu połączenia.

Wspomniane do tej pory aspekty, o ile dobierze się odpowiednie narzędzia, nie
przysparzają problemów w czasie testów. Język TTCN-3\footnote{FIXME link?}
został zaprojektowany aby umożliwić analizę implementacji dowolnych protokołów
zarówno w środowiskach wirtualnych jak i rzeczywistych.

Wydajność transmisji dotyczy między innymi osiąganych przepustowości, reakcji na
zatory, jednoczesnych transmisji, rywalizacji z innymi przepływami i innych
czynników mających wpływ na odczuwalną jakość połączenia.

Test w tym przypadku nie polega zatwierdzeniu lub odrzuceniu pojedynczego
przypadku, a na analizie wielokrotnie powtarzanych eksperymentów i odniesieniu
wyników do teoretycznych założeń.

\subsection{Testy wydajności}

Do pomiaru transferów istnieje sporo narzędzi. Problem z pomiarem
wydajności to konfiguracja środowiska, w którym można wykonywać
powtarzalne pomiary. Trzeba tu więc metod opisujących konfiguracje
prostych sieci, przepustowości łączy, procentu gubionych pakietów, itp,
a także sposobu automatycznego zastosowania zdefiniowanej konfiguracji
na zadanych maszynach/routerach.

\end{document}
